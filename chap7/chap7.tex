\chapter{Explorer interactivement des données : de la simulation à l'empirique (scénarios)}
\label{chap:chap7}
\begin{center}
	{\large Version 2018-XX-XX}
\end{center}
\minitoc

\section{Quels cadres théoriques et méthodologiques pour l'exploration de données multi-dimensionnelles sur le temps long ?}

\subsection{EDA, ESDA, IDA, GDA\dots}
\subsection{Les \textit{visual analytics}}
\subsection{\textit{(geo)Visual Analytics} et données temporelles}

\section{Spécificités des données de simulation : des \textit{big data} comme les autres ?}
\subsection{Des données massives ?}
\subsection{Des données multi-dimensionnelles ?}
\subsection{Des données stables ?}

\section{Vers une validation à l'aide des données de simulation}
\subsection{Confrontation empirique et simulé : l'impossible validation}
\subsection{Le modèle à l'épreuve des scénarios}

\section{Les \textit{(geo)Visual Analytics}, une démarche reproductible et généralisable}
\subsection{Démarche ou méthode ?}
\subsection{Reproductibilité d'une démarche, reproductibilité d'un outil}
\subsection{Appliquer une méthode à d'autres types de données : caractériser la généricité de notre démarche}
\subsection{Vers une utilisation généralisée des \textit{(geo)Visual Analytics} en géographie théorique et quantitative}

% - Comment évaluer/valider le modèle ? -> cross-validation via scénarios
%
%
%## Chapitre 6 : Explorer interactivement des données : de la simulation à l'empirique (scénarios)
%- Positionnement des cadres géomatiques mobilisés : geovis. analytics sur temps long
%- Quelles spécificités des données de simulation ?  (Sim vs Big Data)
%- Comment comparer ces données à des données empiriques ? (Confrontation, en piste)
%- Une démarche d'explo. reproductible et applicable à d'autres données (lesquelles)
%- Vers une utilisation des GeoVis Ana. plus fréquente en GTQ : besoin d'outils *ad-hoc* dédiés à de l'explo