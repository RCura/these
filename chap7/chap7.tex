\chapter{Explorer interactivement des données : de la simulation à l'empirique (scénarios)}
\begin{center}
	{\large Version 2018-XX-XX}
\end{center}
\minitoc

% - Comment évaluer/valider le modèle ? -> cross-validation via scénarios
%
%
%## Chapitre 6 : Explorer interactivement des données : de la simulation à l'empirique (scénarios)
%- Positionnement des cadres géomatiques mobilisés : geovis. analytics sur temps long
%- Quelles spécificités des données de simulation ?  (Sim vs Big Data)
%- Comment comparer ces données à des données empiriques ? (Confrontation, en piste)
%- Une démarche d'explo. reproductible et applicable à d'autres données (lesquelles)
%- Vers une utilisation des GeoVis Ana. plus fréquente en GTQ : besoin d'outils *ad-hoc* dédiés à de l'explo