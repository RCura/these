%\section{Co-construction et accompagnement, un état de l'art}
%\subsection{Modéliser des dynamiques spatiales en contexte interdisciplinaire}
%\subsection{Le cas du \textit{companion modelling}}
%\subsection{L'exploration interactive comme interface disciplinaire}
%
%\section{Un retour sur une expérience de co-construction de modèle}
%\subsection{Cadre de l'expérience : groupe de travail et temporalités}
%\subsection{Conditions de l'expérience : quelle modélisation ?}
%\subsection{Déroulement de l'expérience}
%
%\section{Une méthode de co-construction faite d'allers-retours interdisciplinaires}
%\subsection{Un langage et des outils communs : préférer la co-construction à la prestation}
%\subsection{Modéliser pour explorer, explorer pour modéliser : comment combiner et accumuler les connaissances acquises}
%\subsection{Du particulier au générique : médiation et compromis entre disciplines}
%
%\section{Modéliser en géographe}
%\subsection{Observer un territoire passé : l'impossible terrain}
%\subsection{Modèle spatiaux et modèles spatialisés}
%\subsection{Modéliser en géographe, ne pas modéliser pour les géographes}

\chapter{Modélisation et visualisation comme interfaces disciplinaires}
\label{chap:chap1}
\begin{center}
	{\large Version 2018-XX-XX}	\end{center}
\begin{itemize}
	\item 10/10/2019 : Nouveau plan
\minitoc

\clearpage
\section*{Introduction}
\addcontentsline{toc}{section}{\protect\numberline{}Introduction}

\begin{itemize}
	\item faire parallèles entre modélisation et juridique etc. : on connait le modèle, on l'a créé, et on connait le fonctionnement précis de ses composantes. Mais pas possible de deviner tous les biais que cela engendre de manière interne (bugs, == détournements de loi), et surtout les effets généraux causés par les interactions (politique sociale contre effet par exemple) ==> On a construit le modèle, mais on ne peut en prévoir l'aboutissement.
	\item Parler de l'approche exploratoire : explo de données vs explo de modèle, Ana Sensib vs CAH etc. chap  6
	\item Parler des connaissances expertes : pas de source, manque de connaissance de l'historiographie, donc on fait confiance aux notes et mémoire.
	\item 
\end{itemize}


\section{D'où je viens}

Le travail de recherche présenté dans cette thèse s'inscrit profondément à l'interface entre plusieurs courants disciplinaires liés à l'étude des phénomènes sociaux dans l'espace.
\subsection{Géographie}

\section{Une méthode de co-construction faite d'allers-retours interdisciplinaires}
\subsection{Un langage et des outils communs : préférer la co-construction à la prestation}
\subsection{Modéliser pour explorer, explorer pour modéliser : comment combiner et accumuler les connaissances acquises}
\subsection{Du particulier au générique : médiation et compromis entre disciplines}
\subsection{Géomatique}

\subsection{Modélisation : à la confluence de la GTQ et de la géomatique}
\section{Dans quel contexte s'inscrit ce travail ?}

Chapeau : TransMonDyn et LabEx DynamiTe

\subsection{Modélisation de processus spatiaux}

\subsection{Inscrits dans la longue durée}

\subsection{Dans un contexte fortement interdisciplinaire}


\section{Quel questionnement et évolution ?}

\subsection{Sujet initial}

\subsection{Verrous}

\subsection{Nouvelles questions}

\section{Quelle position ?}

\subsection{Co-construction}

\subsection{Interface}

\subsection{Démarche exploratoire}

\section*{Conclusion}
\addcontentsline{toc}{section}{\protect\numberline{}Conclusion}