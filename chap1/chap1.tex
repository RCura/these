\chapter{Modélisation et visualisation comme interfaces disciplinaires}
\begin{center}
	{\large Version 2018-XX-XX}	\end{center}
\minitoc

\section{Co-construction et accompagnement, un état de l'art}
\subsection{Modéliser des dynamiques spatiales en contexte interdisciplinaire}
\subsection{Le cas du \textit{companion modelling}}
\subsection{L'exploration interactive comme interface disciplinaire}

\section{Un retour sur une expérience de co-construction de modèle}
\subsection{Cadre de l'expérience : groupe de travail et temporalités}
\subsection{Conditions de l'expérience : quelle modélisation ?}
\subsection{Déroulement de l'expérience}

\section{Une méthode de co-construction faite d'allers-retours interdisciplinaires}
\subsection{Un langage et des outils communs : préférer la co-construction à la prestation}
\subsection{Modéliser pour explorer, explorer pour modéliser : comment combiner et accumuler les connaissances acquises}
\subsection{Du particulier au générique : médiation et compromis entre disciplines}

\section{Modéliser en géographe}
\subsection{Observer un territoire passé : l'impossible terrain}
\subsection{Modèle spatiaux et modèles spatialisés}
\subsection{Modéliser en géographe, ne pas modéliser pour les géographes}
