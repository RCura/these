\begin{table}[H]
\centering
\resizebox{1.2\textwidth}{!}{%
{\renewcommand{\arraystretch}{1.2}%
\begin{tabular}{|M{1.6cm}|M{2.25cm}|M{1.7cm}|M{2cm}|M{1.7cm}|M{7cm}|}
\hline
\textbf{Version} & \textbf{Date} & \textbf{Nom original} & \textbf{Nombre de sous-versions} & \textbf{Nombre de \textit{commits}} & \textbf{Changements principaux} \\ \hline
0 & 21/04/2014 & Base & ---\footnotemark[1] & 168 & ---\footnotemark[2] \\ \hline
2 & 13/04/2016 & Base2 & 3 & 31 &
\begin{itemize}[before=\vspace{.5em},after=\vspace{-1em},leftmargin=*]
	\item Mise en place d'une hiérarchie des attracteurs
	\item Ajout de la création de paroisses urbaines
	\item Ajout de la différenciation entre châteaux et gros châteaux
\end{itemize} \\ \hline
3 & 28/08/2016 & Base3 & 4 & 39 & \begin{itemize}[before=\vspace{.5em},after=\vspace{-1em},leftmargin=*]
	\item Réduction des distances de déplacement local
	\item Modification du mécanisme de construction de châteaux
	\item Modification de la logique de calcul de la satisfaction matérielle
\end{itemize} \\ \hline
4 & 25/04/2017 & Base4 & 7 & 59 & \begin{itemize}[before=\vspace{.5em},after=\vspace{-1em},leftmargin=*]
	\item Changement de la répartition initiale des foyers paysans
	\item Changement du mécanisme d’héritage/reconnaissance des agrégats
	\item Changement du mécanisme de définition des pôles
\end{itemize} \\ \hline
5 & 11/06/2018 & v5 & 4 & 60 & \begin{itemize}[before=\vspace{.5em},after=\vspace{-1em},leftmargin=*]
	\item Modification générale de l'ordonnancement des mécanismes
	\item Modification du mécanisme de définition des agrégats
	\item Modification du calcul de satisfaction des foyers paysans
	\item Mise en place de seuils évolutifs pour le déplacement local
\end{itemize} \\ \hline
6 & 09/01/2019 & v6 & \makecell{6 \\ \footnotesize{(22/09/2019)}} & 63 & \begin{itemize}[before=\vspace{.5em},after=\vspace{-1em},leftmargin=*]
	\item Renommage global et refactorisation des paramètres
	\item Suppression des mécanismes de lignage seigneurial
	\item Changement du mécanisme de répartition des nouveaux châteaux
	\item Changement des calculs de probabilité de construction de châteaux
\end{itemize} \\ \hline
\end{tabular}}
}
\caption{Historique des versions de SimFeodal.}
\label{tab:historique-versions-simfeodal}
\end{table}
\footnotetext[1]{Si le code était bien historisé à l'époque, la logique de versionnement n'était pas encore véritablement à l'œuvre.}
\footnotetext[2]{Cette version 0 est marquée par une forte instabilité et par de très fréquents changements.
	On notera d'ailleurs qu'il n'y a pas de version 1.
	En effet, si la version 0 est la première version \og complète\fg{} du modèle, c'est la version 2 qui est la première a avoir été un tant soit peu satisfaisante du point de vue de l'évaluation.}