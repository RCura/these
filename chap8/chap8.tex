\chapter{Retours sur la co-construction et l'exploration d'un modèle en situation d'inter-disciplinarité}
\label{chap:chap8}
\begin{center}
	{\large Version 2018-XX-XX}
\end{center}
\minitoc

\section{Co-construction d'un modèle complexe : un retour d'expérience critique}
\subsection{Accompagnement à la modélisation et modélisation d'accompagnement}
\subsection{Quelle prise en compte de l'hétérogénéité des pratiques et des besoins ?}
\subsection{Se positionner en modélisateur et en thématicien}

\section{Construire et utiliser un modèle, deux approches et positions différentes}
\subsection{Modèle comme finalité, modèle comme apprentissage}
\subsection{Créateur et utilisateur : comment concilier des intérêts antagonistes ?}

\section{Retours sur l'exploration du modèle}
\subsection{Aboutissements de l'exploration}
\subsection{Des gains, certes, mais pour qui ?}
\subsection{Validité de la démarche exploratoire interactive dans un contexte inter-disciplinaire}

%## Chapitre 7 : Retours sur la co-construction et l'exploration d'un modèle en situation d'inter-disciplinarité.
%- Accompagnement modélisation & companion modelling : quelles diffs.
%- Construction du modèle vs utilisation du modèle
%- Comment se positionner ds une modélisation interdisciplinaire
%- A quoi a servi l'explo visuelle ?
