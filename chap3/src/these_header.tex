%\usepackage[utf8]{inputenc}
\usepackage{amsmath}
\usepackage{amsfonts}
\usepackage{amssymb}
\usepackage{graphicx}
\usepackage[left=2.00cm, right=5.00cm, bottom = 2.00cm, top = 2.00cm]{geometry}
\setlength{\marginparwidth}{4cm}

%\usepackage[T1]{fontenc}
\usepackage{fontspec}
\usepackage[french]{babel}
\usepackage[babel=true]{csquotes}

\usepackage{soul}
\usepackage{mdframed}

\usepackage{xargs} % Use more than one optional parameter in a new commands
\usepackage[dvipsnames]{xcolor}
\usepackage[colorinlistoftodos,prependcaption,textsize=tiny]{todonotes}

\usepackage{graphicx}

\usepackage{float}
\usepackage{caption}

\usepackage{fancyhdr}
\pagestyle{fancy}

\renewcommand{\headrulewidth}{0.5pt}
\renewcommand{\footrulewidth}{0pt}
\fancyhead[L]{Chapitre \thechapter}
\fancyhead[C]{}
\fancyhead[R]{\rightmark}


\renewcommand{\thempfootnote}{\arabic{footnote}}%

\usepackage{enumitem}

\usepackage[hyperfootnotes=false]{hyperref}
\usepackage{cleveref}

\usepackage{longtable}
\usepackage{array}



% Encadré pris dans la thèse de Seb Rey
\usepackage[most]{tcolorbox}
\newtcbtheorem[number within=chapter]{encadre}{Encadré}{   
	outer arc=0pt,
	arc=0pt,
	breakable,
	enhanced,
	colback=white,
	colframe=black,
	colbacktitle=white,
	titlerule=0pt,
	fonttitle=\normalcolor\itshape}{enc}

\crefformat{tcb@cnt@encadre}{encadré~#2#1#3}
\Crefformat{tcb@cnt@encadre}{Encadré~#2#1#3}

% ------------------------------------- %
% ---------- COMMANDES PERSO ---------- %
% ------------------------------------- %


% ###### CREATION D'ENCADRES ###### %

%
%\usepackage{newfloat}
%\DeclareFloatingEnvironment[name=Encadré,
%listname={Liste des encadrés},
%within=chapter]{encadreFlot}
%
%\captionsetup[encadreFlot]{labelfont=sc}
%
%\newenvironment{encadre}[1][!htbp]{
%	\begin{encadreFlot}[#1]
%		\renewcommand{\thefootnote}{\textit{\alph{footnote}}}
%		\begin{mdframed}[
%			footnoteinside=false,								innertopmargin=10pt,linecolor=black!20,
%			linewidth=2pt,topline=true,innerrightmargin=10pt,innerleftmargin=10pt]
%		}{\end{mdframed}
%	\renewcommand{\thefootnote}{\alph{footnote}}
%	\end{encadreFlot}
%}
%
%\crefname{encadreFlot}{encadré}{encadrés}
%\Crefname{encadreFlot}{Encadré}{Encadrés}

% Exemple d'encadré
% \begin{encadre}
% \caption{Test\label{enc:bar}}
% \end{encadre}
% Appeler en fin de manuscrit avec
%  \listofmyencadres



% Ancien encadré plus joli mais buggé


%\DeclareFloatingEnvironment[fileext=frm,
%placement={htb},
%listname=Liste des encadrés,
%within=chapter,
%name=BLOB
%]{myencadre}
%
%
%\newcounter{encadre}[chapter]

%\newenvironment{encadre}[1]{%
%	\stepcounter{encadre}
%	\mdfsetup{%
%		frametitle={
%			\tikz[baseline=(current bounding box.east),outer sep=0pt]
%			\node[anchor=east,rounded rectangle,
%			rounded rectangle west arc = 0pt,
%			fill=black!20]
%			{\strut Encadré \thechapter.\arabic{encadre}: \captionlistentry{#1}{#1}};},
%		footnoteinside=false,
%		innertopmargin=10pt,linecolor=black!20,
%		linewidth=2pt,topline=true,
%		innerrightmargin=10pt,innerleftmargin=20pt,
%		frametitleaboveskip=\dimexpr-\ht\strutbox\relax
%	}
%	\begin{myencadre}[!h]\begin{mdframed}[]\relax%
%		}{\end{mdframed}\end{myencadre}}
%
%\crefname{myencadre}{encadré}{encadrés}
%\Crefname{myencadre}{Encadré}{Encadrés}
%
% ------------------------------------- %

% ###### TODO INLINE ###### %
\newcommandx{\unsure}[2][1=]{\todo[linecolor=red,backgroundcolor=red!25,bordercolor=red,#1]{#2}}

\newcommandx{\unsurec}[2][1=]{%
	\colorbox{red!25}{#2}\todo[linecolor=red,backgroundcolor=red!25,bordercolor=red,#1]%
}

\newcommandx{\change}[2][1=]{\todo[linecolor=blue,backgroundcolor=blue!25,bordercolor=blue,#1]{#2}}
\newcommandx{\changec}[2][1=]{%
	\colorbox{blue!25}{#2}\todo[linecolor=blue,backgroundcolor=blue!25,bordercolor=blue,#1]%
}

\newcommandx{\info}[2][1=]{\todo[linecolor=OliveGreen,backgroundcolor=OliveGreen!25,bordercolor=OliveGreen,#1]{#2}}
\newcommandx{\infoc}[2][1=]{%
	\colorbox{OliveGreen!25}{#2}\todo[linecolor=OliveGreen,backgroundcolor=OliveGreen!25,bordercolor=OliveGreen,#1%
}


\newcommandx{\improvement}[2][1=]{\todo[linecolor=Plum,backgroundcolor=Plum!25,bordercolor=Plum,#1]{#2}}
\newcommandx{\improvementc}[2][1=]{%
	\colorbox{Plum!25}{#2}\todo[linecolor=Plum,backgroundcolor=Plum!25,bordercolor=Plum,#1]%
}

% ###### FIXME INLINE ###### %
\newcommandx{\fixref}[1]{#1}