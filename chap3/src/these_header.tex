%\usepackage[utf8]{inputenc}
\usepackage{amsmath}
\usepackage{amsfonts}
\usepackage{amssymb}
\usepackage{graphicx}
\usepackage[left=2.00cm, right=5.00cm, bottom = 2.00cm, top = 2.00cm]{geometry}
\setlength{\marginparwidth}{4cm}

%\usepackage[T1]{fontenc}
\usepackage{fontspec}
\usepackage[french]{babel}
\usepackage[babel=true]{csquotes}

\usepackage{soul}
\usepackage{mdframed}

\usepackage{xargs} % Use more than one optional parameter in a new commands
\usepackage[dvipsnames]{xcolor}
\usepackage[colorinlistoftodos,prependcaption,textsize=tiny]{todonotes}

\usepackage{graphicx}

\usepackage{float}
\usepackage{caption}

\usepackage{fancyhdr}
\pagestyle{fancy}

\renewcommand{\headrulewidth}{0.5pt}
\renewcommand{\footrulewidth}{0pt}
\fancyhead[L]{Chapitre \thechapter}
\fancyhead[C]{}
\fancyhead[R]{\rightmark}


\renewcommand{\thempfootnote}{\arabic{footnote}}%

\usepackage{enumitem}

\usepackage[hyperfootnotes=false]{hyperref}
\usepackage{cleveref}

\usepackage{longtable}
\usepackage{array}

\usepackage[style=authoryear-comp,
hyperref,
backend=bibtex,
isbn=false,
doi=true,
url=true,
date=year]{biblatex}


% ------------------------------------- %
% ---------- COMMANDES PERSO ---------- %
% ------------------------------------- %

% ###### CREATION D'ENCADRES ###### %

% Encadré pris dans la thèse de Seb Rey
\usepackage[most]{tcolorbox}
\newtcbtheorem[number within=chapter]{encadre}{Encadré}{   
	outer arc=0pt,
	arc=0pt,
	breakable,
	enhanced,
	colback=white,
	colframe=black,
	colbacktitle=white,
	titlerule=0pt,
	fonttitle=\normalcolor\itshape}{enc}

\crefformat{tcb@cnt@encadre}{encadré~#2#1#3}
\Crefformat{tcb@cnt@encadre}{Encadré~#2#1#3}

% ###### COMMENTAIRES ###### %

% Commentaire en marge
\newcommandx{\change}[2][1=]{\todo[linecolor=blue,backgroundcolor=blue!25,bordercolor=blue,#1]{#2}}
% Commentaire en marge + surlignement texte
\newcommandx{\changec}[2][1=]{%
	\colorbox{blue!25}{#2}\todo[linecolor=blue,backgroundcolor=blue!25,bordercolor=blue,#1]%
}
% Highlight
\newcommandx{\fixref}[1]{\hl{#1}}