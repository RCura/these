\begin{table}[H]
	\centering
	{\renewcommand{\arraystretch}{1.2}%
	\begin{tabular}{|M{2.15cm}|M{4cm}|m{5.25cm}|m{1.25cm}|}
		\hline
		\textbf{Sous-mécanisme} & \textbf{Paramètre} & \textbf{Intitulé dans le modèle implémenté} & \textbf{Valeur} \\ \hline
		\makecell{Création\\du\\ \og monde\fg{} \\ du modèle} & Dimension & taille\_cote\_monde & 80 km \\ \hline
		\multirow{7}{*}{\makecell{Génération\\des\\Foyers\\Paysans\\(FP)}} & Nombre total de FP & init\_nb\_total\_fp & 4000 \\ \cline{2-4} 
		& Nombre de petites villes & init\_nb\_agglos & 8 \\ \cline{2-4} 
		& Nombre de FP par petite ville & init\_nb\_fp\_agglo & 30 \\ \cline{2-4} 
		& Nombre de villages & init\_nb\_villages & 20 \\ \cline{2-4} 
		& Nombre de FP par village & init\_nb\_fp\_village & 10 \\ \cline{2-4} 
		& Distance d'agrégation des FP & distance\_detection\_agregat & 100 m \\ \cline{2-4} 
		& \makecell{Taux de FP \\ \og dépendants\fg{}\\(ie. non mobiles)} & proba\_fp\_dependant & 20\% \\ \hline
		\multirow{2}{*}{\makecell{Génération\\des\\Églises}} & Nombre total d'églises & init\_nb\_eglises & 150 \\ \cline{2-4} 
		& (dont) Nombre d'églises paroissiales & init\_nb\_eglises\_paroissiales & 50 \\ \hline
		\multirow{3}{*}{\makecell{Génération\\des\\Seigneurs}} & Nombre de Grands Seigneurs & init\_nb\_gs & 2 \\ \cline{2-4} 
		& \makecell{Puissance relative\\des Grands Seigneurs\footnotemark} & \makecell{puissance\_grand\_seigneur$1$ \\ \textelp{} \\ puissance\_grand\_seigneur$N$} & 50\% \\ \cline{2-4} 
		& Nombre de Petits Seigneurs & init\_nb\_ps & 18 \\ \hline
	\end{tabular}}
	\caption{Paramètres permettant de contrôler l'initialisation du monde de \simfeodal{}.}
\label{tab:params-initiaux}
\end{table}
\footnotetext{
	Ces paramètres sont plus spécifiques que les autres présentés ici et n'ont pas été introduits avant.
	Ils conditionnent la part des loyers que les grands seigneurs collectent.
	Voir la \cref{fig:prelevement-fonciers}, dans la \cref{sssec:collecte-droits}, p.~\pageref{fig:prelevement-fonciers}.
}