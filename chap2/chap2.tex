% !TEX root = ../These_Robin_Master.tex
\chapter{Formaliser connaissances et hypothèses, vers un modèle de simulation co-construit : SimFeodal}
\label{chap:chap2}
\begin{center}
	{\large Version \hl{2019-04-12}}
\end{center}

\begin{itemize}
	\item Avril 2019 : Nouveau départ pour le chapitre
\end{itemize}
\setcounter{minitocdepth}{2}
\minitoc

\textbf{Code couleur}
\begin{itemize}
	\item Texte écrit pour cette version du chapitre
	\item {\redroman Texte publié dans le chapitre de Peupler la Terre}
	\item {\blueroman Modifications dans le texte du chapitre de Peupler la Terre}
\end{itemize}

\clearpage

\addcontentsline{toc}{section}{\textit{\textmd{Avant-propos}}}
\begin{mdframed}[backgroundcolor=gray!10,footnoteinside=false]
	\textbf{Avant-propos}\\
Le présent chapitre décrit un modèle, SimFeodal, qui est une œuvre profondément collective et interdisciplinaire.
La paternité de ce modèle est ainsi à attribuer à l'auteur de ces lignes autant qu'à l'ensemble des co-concepteurs du modèle :
\begin{itemize}
	\item Cécile \textsc{Tannier}, UMR 6049 ThéMA -- Besançon\\
	Géographe et modélisatrice, Directrice de Recherche, CNRS
	\item Samuel \textsc{Leturcq}, UMR 7324 CITERES-LAT -- Tours\\
	Historien, Maître de Conférence, Université François Rabelais
	\item Elisabeth \textsc{Zadora-Rio}, UMR 7324 CITERES-LAT -- Tours\\
	Archéologue, Directrice de Recherche émérite, CNRS
	\item Élisabeth \textsc{Lorans}, UMR 7324 CITERES-LAT -- Tours\\
	Archéologue, Professeure, Université François-Rabelais
	\item Xavier \textsc{Rodier}, UMR 7324 CITERES-LAT -- Tours\\
	Archéologue, Ingénieur de Recherche HDR, CNRS

\end{itemize}

Ce chapitre de thèse constitue une reprise, individuelle et largement modifiée et retravaillée, d'un chapitre -- \citetitle{cura_transition_2017} \autocite{cura_transition_2017} -- de l'ouvrage collectif \og Peupler la terre\fg{} \autocite{sanders2018peupler} issu du projet TransMonDyn\footnotemark.
Dans cette thèse, SimFeodal est présenté dans une version différente de celle de l'ouvrage collectif, et de nombreux mécanismes sont par exemple considérablement simplifiés.
Les fondements du modèle, toutefois, restent très largement identiques entre ces deux versions, et à ce titre, ce chapitre de thèse reprend parfois des passages entiers du chapitre de l'ouvrage collectif.
Dans ces moments, nous avons préféré ne pas les identifier en tant que tel, notamment car l'entremêlement de modifications apportées rendrait difficile la lecture.

En matière de forme, notons que contrairement au chapitre publié, le modèle SimFeodal est ici présenté en suivant le protocole de description \og ODD\fg{} (\textit{Overview, Design concepts, and Details}) \autocite{grimm_odd_2010}, dans sa formulation la plus récente (\cite{grimm_documenting_2017}, voir \cref{tab:proto-ODD}).
SimFeodal ne se prête pas à toutes les catégories identifiées par les auteurs de ce standard, et celui-ci n'est de plus pas pensé pour une description aussi détaillée du modèle\footnotemark, mais nous pensons tout de même que le suivi de ce standard d'adoption permettra d'augmenter la reproductibilité de SimFeodal.
Pour cette même raison, notons que l'implémentation du modèle, son historique ainsi que les différentes descriptions techniques sont disponibles dans le dépot de versionnement de SimFeodal :
\begin{center}
	\href{https://github.com/SimFeodal/SimFeodal}{https://github.com/SimFeodal/SimFeodal}
\end{center} 
\end{mdframed}
\footnotetext[1]{
Projet ANR (ANR-10-BLAN-1805), coordonné par Lena \textsc{Sanders}, entre 2011 et 2014.
\href{www.transmondyn.parisgeo.cnrs.fr}{www.transmondyn.parisgeo.cnrs.fr}
}
\footnotetext[2]{
\hl{Une description plus courte, plus proche des descriptions ODD classiques, est disponible en ligne.}
}


\section*{Introduction}
\label{sec:chap2-intro}
\addcontentsline{toc}{section}{Introduction}

{\redroman
	La question de l'émergence de la société féodale en Occident est au cœur d'un débat historique ancien.
	Depuis le XVIIIe siècle, les penseurs cherchent à comprendre le fonctionnement de la société médiévale et à cerner ses fondements.
	Les archives sont continûment explorées pour comprendre isolément et précisément les multiples facteurs à l'œuvre dans les processus qui ont fait émerger une société dite « féodale » dans le courant des Xe-XIe siècles.
	Cette compréhension se heurte toutefois à la très grande complexité de ces processus, qui peuvent varier chronologiquement, mais aussi présenter des nuances infinies en fonction des zones étudiées.
	Ces difficultés sont encore amplifiées par l'accès aux données, très variable selon l'état de la documentation, soumise aux aléas de la conservation ; d'une manière générale, les historiens des temps féodaux travaillent sur des documents rares et lacunaires, vestiges d'une société fondamentalement portée par l'oralité.
	
	Depuis une quarantaine d'années, l'afflux massif de données de fouilles issues du développement de l'archéologie préventive a permis de renouveler et enrichir ces débats.
	Les sources textuelles, qui apportent un éclairage plutôt normatif de la société, peuvent désormais être confrontées à des sources matérielles propres à mieux cerner les dimensions pratiques.
	Toutefois, cette complémentarité des approches textuelles et matérielles, loin de simplifier les questionnements portant sur la société féodale, les a encore complexifiés en mettant en évidence des aspects anthropologiques et des différenciations géographiques jusqu'alors sous-estimés.
	Le débat s'en est trouvé vivifié, se focalisant désormais sur la question de l'occupation de l'espace, considéré comme un marqueur efficace des transformations sociales.
	L'émiettement et la dissémination des pouvoirs, dont témoigne la multiplication des châteaux (seigneuries châtelaines), se font concomitamment à l'apparition d'un réseau très structuré d'encadrement religieux (paroissialisation de la société), tandis que se fixe de manière définitive un système de peuplement fondé sur un maillage villageois, cœur d'une vie communautaire active.
	
	C'est donc autour de l'articulation de ces trois éléments fondamentaux de la société féodale (châteaux, églises paroissiales, villages) que portent aujourd'hui analyses et théories.
	Fixation, polarisation et hiérarchisation des centres de peuplement sont désormais les grands processus sociaux examinés à la loupe pour aborder la société médiévale.
	Les historiens médiévistes analysent l'« encellulement » de la société \autocite{fossier_enfance_1982}, pistant d'une part les rôles polarisateurs du château (phénomène d'\textit{incastellamento} \autocite{toubert_les_1973}) et de l'église paroissiale accompagnée de son cimetière, considérés comme points de ralliement des populations paysannes, et d'autre part les manières dont les populations organisent collectivement les espaces de production (terroir villageois) pour assurer une répartition équilibrée des ressources.
	
	Dans ce contexte, la période 800-1100 est habituellement considérée comme une période de transition, durant laquelle la société féodale se structure, certains évoquant la « révolution de l'an Mil »\autocite{fossier_enfance_1982}, tandis que d'autres tempèrent en parlant de « révélation de l'an Mil » \autocite{barthelemy_societe_1993}(« révélation » par l'augmentation en quantité et en qualité de la documentation textuelle).
	Les hypothèses sont ainsi nombreuses, et il est difficile de trancher en faveur de l'une ou l'autre, tant l'articulation des facteurs sociaux, politiques, institutionnels, économiques et culturels est complexe.
}

\section[Objectif du modèle SimFeodal --  \textit{Purpose}]{Objectif du modèle SimFeodal\protect\newline \large{\textit{Purpose}}}

{\redroman
	Dans le cadre de l'ANR TransMonDyn, l'objectif, pour la transition des années 800-1100, est d'étudier les processus à l'œuvre dans la dynamique de fixation, polarisation et hiérarchisation de l'habitat rural.
	L'approche est résolument géographique ; ce sont les implications spatiales des changements sociaux qui sont au cœur de l'étude.
	La modélisation ne porte pas sur les transformations politiques et sociales elles-mêmes, mais sur leur impact sur le système de peuplement.
	Le cœur du questionnement réside dans l'examen de la combinaison des facteurs ayant permis, entre 800 et 1100, la formation d'agrégats de foyers paysans dans une forme hiérarchisée et durable, polarisés par des châteaux ou des églises.
	Il s'agit d'analyser, par la modélisation et la simulation informatique, les conditions d'émergence du maillage villageois.
}


\section[Entités et échelles -- \textit{Entities, state variables, and scales}]{Entités et échelles\protect\newline \large{\textit{Entities, state variables, and scales}}}

\subsection{Entités}

Descriptif + tableau

\subsection{Échelles spatiales et temporelles}

\subsubsection{Résolution et échelle spatiale}

Descriptif : espace continu, choix de la taille du monde, paramétrable
+ schema

\subsubsection{Granularité temporelle}

Descriptif : choix des dates de début et de fin (+ paramètres ?) + logique de on prolonge pour observer. + Logique de temps discret : repère plus que correspondance exacte
+ frise

\section[Fonctionnement général -- \textit{Process overview and schedulling}]{Fonctionnement général\protect\newline \large{\textit{Process overview and schedulling}}}

\subsection{Ordonnancement général}


\section[Design -- \textit{Design concepts}]{Design \protect\newline \large{\textit{Design concepts}}}

\subsection{Basic principles}
\subsection{Emergence}
\subsection{Adaptation}
\subsection{Objectives}
\subsection{Learning}
\subsection{Prediction}
\subsection{Sensing}
\subsection{Interaction}
\subsection{Stochasticity}
\subsection{Collectives}
\subsection{Observation}


\section[Situation initiale -- \textit{Details - Initialisation}]{Situation initiale\protect\newline \large{\textit{Details - Initialisation}}}

\subsection{Une situation initiale théorique et endogène}
\subsection{Paramètres d'initialisation}

\section[Données en entrée -- \textit{Input data}]{Données en entrée\protect\newline \large{\textit{Input data}}}

\section[Mécanismes spécifiques -- \textit{Submodels}]{Mécanismes spécifiques\protect\newline \large{\textit{Submodels}}}

\subsection{Introduction} : Pas tout résumé ici

\subsection{Mécanismes globals}
	\subsubsection{Identification des agrégats}
	\subsubsection{Identification des pôles}
	\subsubsection{Identification des églises paroissiales}
	\subsubsection{Création et promotion d'églises paroissiales}

\subsection{Foyers paysans}
	\subsubsection{Renouvellement des foyers paysans}
	\subsubsection{Satisfaction et modèles gravitaires}
	\subsubsection{Déplacement des foyers paysans}

\subsection{Seigneurs}
	\subsubsection{Construction de châteaux}
	\subsubsection{Dons de châteaux}
	\subsubsection{Création et dons de droits}
	\subsubsection{Prélèvement des loyers}
	\subsubsection{Prélèvement des droits}

\printbibliography[title={Références}]
