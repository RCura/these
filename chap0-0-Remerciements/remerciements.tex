% !TEX root = ../These_Robin_Master.tex
%\newgeometry{top=1.3cm, bottom=.9cm, left=2cm, right=2cm}
\chapter*{Remerciements}
%\vspace{-1em}
%0 - Intro
Les premières lignes d'un manuscrit de thèse sont le plus souvent dédiées à des dédicaces, des épigraphes ou encore à des mentions administratives imposées.
Je suis ici particulièrement heureux de pouvoir finir l'écriture de ce manuscrit en l'introduisant par l'expression de ma gratitude envers toutes celles et ceux qui ont permis à cette thèse de débuter et de la faire aboutir.

Avant tout, je tiens à remercier celles qui ont encadré ce travail.
%1 - Lena
Lena, tu as su m'accompagner dans la conception, la construction et l'évaluation -- interne autant qu'externe -- de ce travail, toujours avec une bienveillance sincère, une rigueur aigüe et une compréhension fine de la posture d'encadrement, évolutive, dont j'avais besoin pour que ce travail aboutisse d'une façon heureuse.
Pour cette habileté humaine et scientifique, qui ont toutes deux participé à et guidé ma (co-)construction de jeune chercheur, un \og incommensurable\fg{} merci à toi.
%2 - Anne
Anne, par tes remarques et questions -- faussement ingénues -- sur l'objectif de la modélisation, de la simulation et, plus globalement sur le lien entre ce que je voulais montrer et la manière de l'appuyer, tu m'as poussé à expliciter mes hypothèses et le sens de mes idées.
Pour l'éclaircissement, dans le manuscrit et dans mon esprit, qui en résulte, je te remercie sincèrement.

%3 - Jury
J'ai la chance et l'honneur d'être évalué par des chercheuses et chercheurs qui m'inspirent chacun à leur manière.
Qu'ils le sachent ou non, le travail qu'ils ont accepté de discuter a largement été influencé par la lecture de leurs travaux, l'écoute de leurs présentations, les discussions tenues avec eux ou encore les postures qu'ils adoptent et défendent.
Pour tout cela, je tiens à remercier Sidonie Christophe, Jean-Daniel Fekete, Didier Josselin, Laure Nuninger et Thomas Thévenin d'avoir accepté de composer le jury de cette thèse et de se prêter à l'exercice de son évaluation.

%4 - Hélène
Depuis le master et jusqu'à aujourd'hui, en passant par ElementR, TransMonDyn, les formations communes, le GdR MAGIS et enfin cette école \og d'été\fg{} GéoViz 2018, j'ai toujours pu compter sur ta générosité, tes conseils, ta pédagogie, ton sens du collectif, ta curiosité, ta motivation, et -- tu pardonneras à ma sincérité cette expression -- ton mentorat.
Hélène, merci pour ces années de collaboration -- passées et sûrement à venir.
\clearpage
%5 - Cécile, Samuel, Elisabeth, mais aussi Élisabeth Lorans et Xavier Rodier
Il me faut maintenant remercier ceux sans qui cette expérience de modélisation collective n'existerait pas, ou n'aurait certainement pas été aussi stimulante.
Cécile, merci d'avoir animé ce groupe \og transition 8\fg{} pendant toutes ces années, en introduisant sans cesse de nouveaux objectifs pour en assurer la continuité.
Merci aussi pour les sessions de travail intensives, à Paris, à Besançon ou à distance, qui ont toujours relancé la co-construction de SimFeodal quand les nombreux autres projets de chacun le laissaient de côté trop longtemps.
Merci enfin pour ta simplicité, ta motivation, pour l'écoute dont tu fais preuve vis-à-vis des préoccupations de chacun, et pour ton amitié.
Un merci non moindre à Elisabeth Z.-R. et Samuel, qui avez su distiller vos immenses connaissances historico-archéologiques à chaque échange, avec une incroyable patience et une certaine abnégation quand on vous demandait pour la énième fois d'essayer de quantifier des éléments qui ne peuvent l'être.
Votre dévouement m'impressionne, de même que votre investissement dans ce projet de longue haleine qui ne constitue pourtant qu'une petite facette de vos recherches.
Je tiens enfin à remercier Élisabeth L. et Xavier, pour les débuts de cette aventure et la manière dont ils l'ont marquée.
J'espère que les événements futurs autour de SimFeodal constitueront des occasions d'en discuter les évolutions.
À toutes et tous, merci pour cette belle dynamique collective et horizontale.
%

% intro bis
La recherche est bien plus agréable à mes yeux -- j'espère que ce travail l'illustre -- quand elle est menée de manière collective.
Je tiens donc à remercier les nombreuses personnes -- qui rendent la concision difficile --  avec l'aide desquelles j'ai pu forger mes idées, développer mes connaissances, et découvrir des approches différentes.

%6 - Julie et Lucie
En tout premier lieu, mes remerciements vont à Julie et Lucie.
Notre petit trio s'est créé au sein de TransMonDyn, a été renforcé par le LabEx, et je ne conçois plus de développer une idée ou un texte sans faire appel à votre avis.
Tout au long de ces années de thèse, vous m'avez accompagné dans tous les moments -- difficiles et joyeux --, scientifiquement et personnellement, et vos approches sensibles de l'interdisciplinarité m'ont apporté bien plus que je ne pourrais le dire dans ces quelques lignes.
Merci de votre exigence intellectuelle commune -- et à Lena pour ce \og formatage\fg{} qu'elle vous (et j'espère nous) a légué --, de votre disponibilité et de votre générosité, en travail comme en amitié, sans faille.
Si ces années de thèse, même dans les derniers jours, n'ont jamais été solitaires, c'est avant tout grâce à vous.

%7 - Arrivée recherche :
%- Renaud et Antonine
%- Thomas, Seb, Clara, Denise et Arnaud
Merci à Renaud et Antonine d'avoir accompagné mes premiers pas dans la recherche et de m'avoir encouragé à persister, même quand rien n'allait.
Merci à Thomas, Sébastien et Clara de m'avoir fait entrer au labo et de m'avoir continuellement motivé à y prolonger mon séjour.
Mes thématiques de recherche vous doivent beaucoup, de même que les approches que je défends et celles dont je m'écarte.
Clara s'est lancée dans d'autres aventures, mais Thomas et Sébastien, j'ai hâte de pouvoir désormais initier d'autres travaux avec vous, en confiance et en amitié.
Cette arrivée dans l'UMR a aussi été l'occasion de rencontrer Denise et Arnaud.
Que ce soit dans le cadre de vos projets de recherche, au milieu d'un couloir, en colloque ou lors d'une collaboration, c'est beaucoup à votre contact que j'ai découvert la géographie quantitative, la modélisation, la simulation, etc.
Merci infiniment de votre accessibilité et du plaisir évident et communicatif que vous prenez à former.
\clearpage

%7 - TransMonDyn, GeoDiverCity, LabEx
Merci aux acteurs des projets auxquels j'ai eu la chance de participer : 
à Lena, pour TransMonDyn, et à tous les membres de ce formidable projet ; à Denise, pour GeoDiverCity, et aux acteurs du groupe \og données\fg{} autant que du groupe \og simulation\fg{} ; aux membres du groupe \og Temps long\fg{} du LabEx, et en particulier à Nicolas dont je me sens privilégié d'avoir fait la rencontre à cette occasion, et à celles avec qui j'ai travaillé plus étroitement et qui m'ont beaucoup apporté : Marie-Vic et Clara~F.
Merci au groupe ElementR, au GDR MAGIS et notamment à l'A.P. GéoViz et aux organisateurs de son école : Hélène, Paule-Annick, Marlène et Sidonie.

%8 - Admin : Labex, Candice, Saber, Martine, Véronique, Stéphanie, Yonathan
Merci à ceux qui ont toujours facilité mon parcours administratif et matériel, notamment Sophie, Anaïs, Farouk, Pauline et Pierre au LabEx DynamiTe, et Candice à l'école doctorale.
Merci aux \og cadres\fg{} de l'UMR qui rendent tout si simple pour les autres et concourent à ce que ce laboratoire fonctionne si bien.
Martine, Saber, Véronique, Liliane, Stéphanie et Yonathan, vous m'avez rendu d'innombrables services, avec beaucoup de patience, de disponibilité et de bonne humeur : merci !

%9 - Labo
En près de 9 ans, j'ai eu la chance de rencontrer énormément de personnes formidables, qui toutes ensemble composent un cadre de travail extraordinaire et stimulant, l'UMR Géographie-cités.
Merci à ceux qui m'en ont fait découvrir le fonctionnement (Julie V., Antoine, Hélène, Céline, etc.), à ceux qui ont toujours répondu à mes questions, même naïves, au détour du couloir de la rue du Four (Hélène, Antoine, Nadine, Arnaud, Anne B., Sophie, Clarisse, etc.), à ceux qui m'ont fait part de tant de remarques utiles, en réunions d'équipe ou ailleurs (Denise, Éric, Claude, François, etc.).
Merci aux doctorants, ingénieurs et jeunes docteurs de l'époque qui m'ont accueilli de manière si conviviale et m'ont poussé à y rester : Thomas, Sébastien, Clara, Guilhain, Hadrien, Marion, Stavros, Florent, Julie F.-C., Delphine, Charlène, Dimitra, Antonin, Caroline, Jean-Baptiste, Sylvain, Elfie, Elodie et Laurent.
Merci à ceux qui ont découvert le labo en même temps que moi et qui ont constitué un soutien tangible dans ces premiers mois de découverte : Clem, Solène, Zoé, Etienne, Romain, Mathieu, Paul, puis Olivier, Sylvestre, Brenda, Julie et Marion P.
Merci aux anciens doctorants de \og ma génération\fg{} : Lucie, Dorian, Damien, Thibault, avec moi finit ce cycle décidément porté sur les thèses longues !
Merci à tous ces lamibos déjà cités, et à ceux qui ont suivi, pour les nombreuses discussions, rue du Four, en 402 à l'IG, au Coolin, au Bar-a-Stavros, au Relais-Fac (merci Matthieu !), au Tennessee, et maintenant au Front Pop' : Odile, Ryma, Natalia, Eugenia, Juste, Dilruba, Ulysse, Matthieu, Mattia, Julien, Constance, Paul G., Anaïs D., Anne-Cécile, Aurélie, Milan, Justin, Joséphine, Marion A., Clément, Marie, Thomas R., Mathilde, Fanny et tous ceux qui prendront la suite.

%10 - Labo++
Merci à ceux qui ont toujours porté des projets collectifs et rendu l'atmosphère de la rue du Four si vivifiante : Hélène, Hadri, Thomas, Marion, Seb, Clem, Solène, Brenda, Julie, Lucie.
Et à ceux qui le reproduisent sur notre nouveau Campus : Anne-Cécile et Matthieu, les \og chefs-de-Condorcet\fg{}, Marion A. pour ses quizz, et tous ceux qui y participent.
Un merci particulier aux lamibos canal historique tendance guacamole, qui se reconnaitront, pour toutes les soirées où les rires ont chassé les \og rayons de trou noir\fg{}.
Un grand merci aussi au groupe Bretagne, pour votre support -- scientifique, émotionnel, et même logistique ces dernières semaines (merci la coloc !) -- et pour votre amitié patiente, même face à mes provocations et à ma mauvaise foi absolue.
\clearpage

%11 - Lamibos
Un dernier merci aux collègues qui m'ont aidé à finaliser ce manuscrit, en relisant ces longues pages pour y détecter les coquilles, et corriger des tournures, et débattre de leur contenu.
Merci Clarisse, Julien, Julie, Thomas, Clémentine, Sébastien, Lucie, Paul C. et Paul G., Thibault, Anne-Cécile, Matthieu, Julie, Lucie et encore Julie et Lucie.
Je vous en suis profondément reconnaissant.

%12 - Carthagéo
Merci aux amis du master, Emeric, Thierry \& Gaëlle, Lucie, Mathieu et Sandy, sources inépuisables de chamailleries et de railleries, et dont les visions actuelles de ce que sont effectivement la géomatique et la cartographie constituent toujours un immense bol d'air frais face à certains usages académiques.

%13 - Amis
Merci aussi aux vieux copains, de plus de 10 ou 15 ans quand ce n'est pas de toujours (Aurel), qui constituent ma famille de cœur aujourd'hui.
Pour la branche Mélinée, Brozio, Bubur, Noirot, Miki, Eva, Manon, Ikram et Andy, et pour la branche Rumont (merci pour tout Granny), Pauline, Alex, Lola, Roman, Lucile, H, Robin, Gaspard et Violette.
Bienvenue aux derniers arrivants de ces joyeuses tribus, Jim, Léon, Leïla et Marcus.
L'avancée de la thèse m'a rendu inversement disponible, mais je compte bien me rattraper de tout ce temps -- pas vraiment -- perdu.

%14 - Famille
Merci enfin aux membres de ma famille, actuelle et en cours de co-construction.
Pour votre gigantesque et incessant soutien, pour votre veille efficace du baromètre de mon moral, pour vos motivations et votre habileté et efforts à faire en sorte que je n'ai à me soucier que de cette thèse, ces derniers mois mais aussi ces dernières années.
Brigitte, Francis, je ne saurai jamais comment vous exprimer ma gratitude, pour tout, et je ne peux qu'espérer que vous en connaissez l'ampleur, tout autant que celle de mon amour.
Rémi, Romane, vous avez ouvert la voie, merci de m'avoir sans cesse encouragé, conseillé, et gâté, notamment chez vous où j'ai pu accoucher d'un chapitre difficile.
Il me tarde de vous retrouver, à Boston ou ici.
Coline, merci pour ta prévenance, ta spontanéité, et ton comique de répétition -- même dans les moments difficile -- que je partage.
J'espère que ton amour des mots me pardonnera les sans doute nombreuses erreurs qui parsèment encore ce manuscrit, ce n'est pas faute d'avoir écouté tes -- encore plus nombreuses -- règles et d'avoir tenté de les mettre en pratique.
Julie, c'est au cours de cette thèse que j'ai eu l'occasion de te rencontrer, de te découvrir, et petit à petit, de t'aimer.
Sans ton apport, tes éclairages, ta culture et ta curiosité, cet ouvrage pour lequel tu m'as toujours inspiré et encouragé n'aurait sans doute pas été complet, et ma vie quotidienne ne l'aurait pas non plus été autant.
Merci pour ton ouverture, ta sensibilité à fleur de peau, ta \og non-geekerie\fg{}, ta passion et ta douceur qui sont autant de raisons, s'il en fallait vraiment, d'avoir aimé chacun des moments passés à tes côtés sur ce début de temps long.
