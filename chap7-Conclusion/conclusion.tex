% !TEX root = ../These_Robin_Master.tex
\graphicspath{{chap7-Conclusion/}}

\FancyNotChapter{Conclusion générale}
\chapter*{Conclusion générale}
\label{chap:conclu}
\mtcaddchapter[Conclusion générale]
%\vspace*{-5em}
%\begin{center}
%	{\large Version \hl{2020-01-03}}
%\end{center}
%\vspace*{-1em}
%\begin{itemize}
%	\item 14/12/2019 : début réflexion plan
%	\item 15/12/2019 : partie synthèse
%	\item 17/12/2019 : fin premier jet Conclusion
%\end{itemize}

Dans cette thèse, nous avons proposé et illustré une approche engagée de modélisation de dynamiques spatiales sur le temps long.
Cette approche s'appuie sur un contexte interdisciplinaire et plaide pour la co-construction de modèles de simulation à base d'agents par le recours à des méthodes qui en facilitent la démarche.
L'approche d'ensemble proposée a été mise en oeuvre sur le cas d'étude de la modélisation des transformations du système de peuplement du Nord-Ouest européen entre 800 et 1100, à l'aide d'un groupe composé de modélisateurs et d'experts thématiciens (archéologues, historiens et géographes).
Cette modélisation a débouché sur la conception et l'implémentation d'un modèle de simulation à base d'agents, \simfeodal{}, qui s'inscrit dans un paradigme descriptif, exploratoire et à visée heuristique.
Afin d'évaluer ce modèle par le biais des données spatio-temporelles massives qu'il génère, nous avons conçu une méthode d'évaluation fondée sur l'analyse graphique d'indicateurs de sortie de simulation :
	l'évaluation visuelle.
Pour faciliter l'emploi de cette méthode d'évaluation, nous avons conçu et développé un outil interactif dédié à l'exploration des sorties du modèle \simfeodal{}, la plateforme \simedb{}.
Cet outil permet à chacun des membres du projet, thématicien comme modélisateur, de mener ses propres explorations des dynamiques simulées par le modèle et ainsi de gagner en compréhension de son comportement, favorisant l'itération entre les hypothèses thématiques et leurs résultats simulés.

Le dernier chapitre a permis d'effectuer un retour réflexif, méthodologique et conceptuel sur le travail mené dans cette expérience de modélisation interdisciplinaire.
Dans cette conclusion, nous reviendrons tout d'abord plus généralement sur la progression globale de cette thèse et sur ce qui lui donne sa cohérence.
Nous présenterons ensuite les nombreuses perspectives qui ont émergé de ce travail, chacune d'entre elles étant liées aux différentes disciplines et sous-disciplines mobilisées.


\subsection*{Synthèse du travail}

Dans le \cref{chap:chap1}, nous décrivons le contexte, personnel et collectif, de mise en œuvre de ce travail.
Cela permet d'en établir le positionnement à l'interface entre la géographie, la géomatique, les sciences historiques, les sciences de la complexité et l'informatique.
À partir de ce positionnement hybride, nous sommes en mesure de spécifier l'objet de la thèse et de justifier le cas d'étude choisi. C'est depuis les spécificités de ce cas d'étude que sont proposées les approches principales défendues dans cette thèse.
Celles-ci reposent, d'une part, sur la co-construction d'un modèle via le paradigme de la simulation à base d'agents et, d'autre part, sur le recours à la comparaison et à l'exploration visuelle de données spatio-temporelles.
Ces deux approches méthodologiques sont proposées car elles permettent d'établir des interfaces entre les disciplines, et visent à faciliter la réalisation collective d'un processus de modélisation des transformations dans les systèmes de peuplements sur le temps long. 

L'approche proposée est mise en œuvre dans le \cref{chap:chap2}.
Dans ce chapitre, nous présentons plus avant le cas d'étude, fondé sur la modélisation des dynamiques de polarisation, de hiérarchisation et de fixation de l'habitat rural entre 800 et 1200 dans la région Touraine.
Nous décrivons ensuite le modèle de simulation à base d'agents que nous avons construit, \simfeodal{}, en suivant le formalisme ODD \autocite{grimm_odd_2010}.
Ce modèle à visée exploratoire suit l'approche KIDS de \textcite{edmonds_kiss_2005}.
Il se classe ainsi dans la catégorie des modèles descriptifs, avec de nombreux types d'agents (foyers paysans, seigneurs, églises paroissiales, pôles d'attraction, etc.) dont les interactions multiscalaires font émerger les dynamiques étudiées.
Le modèle \simfeodal{} constitue dès lors un support de discussion et d'analyse pour notre groupe de travail, contribuant ainsi au rôle heuristique de cette démarche collective et interdisciplinaire.


Le \cref{chap:chap3} présente une synthèse sur l'évaluation des modèles de simulations et propose, à partir de celle-ci, une méthode adaptée à l'évaluation collective et interdisciplinaire de modèles complexes.
Cette méthode, l'\og évaluation visuelle\fg{}, se situe dans le prolongement de la \textit{face validation} et s'appuie sur la création d'\og{}indicateurs de sortie de simulation\fg{} qui sont des représentations synthétiques numériques et graphiques des dynamiques générées par le modèle.
Pour évaluer le modèle, on compare ces indicateurs à des \og ordres de grandeur\fg{} et à des \og formes stylisées\fg{}, c'est-à-dire à des \og indices empiriques\fg{}, selon une grille d'évaluation fixée \textit{a priori}.
Par la réalisation de phases d'évaluation fréquentes et itératives, on procède au \og paramétrage\fg{} du modèle, c'est-à-dire à son amélioration par l'intermédiaire de modifications des mécanismes et d'ajustement des paramètres.
Nous défendons ainsi une approche exploratoire du modèle, guidée par de nombreux allers-retours entre sa construction et son évaluation.
Nous illustrons enfin cette approche par une analyse rétrospective des différentes versions ponctuant l'évolution de \simfeodal{}.

Pour faciliter l'exercice d'évaluation visuelle, nous construisons un outil interactif dédié à l'exploration des données de sortie de \simfeodal{}, la plateforme \simedb{}, que nous présentons dans le \cref{chap:chap4}.
Cette plateforme résulte d'une succession de réponses aux contraintes génériques et spécifiques posées par la nature volumineuse et hétérogène des données permettant la génération des indicateurs de sortie.
Après avoir retracé l'historique des solutions choisies pour faciliter l'évaluation de \simfeodal{} à chaque phase de sa construction, nous justifions le choix d'organisation des données qui en sont issues sous forme de bases de données relationnelle analytique.
Une telle structuration est nécessaire à une capacité d'interrogation robuste, performante et évolutive de ces données.
La plateforme \simedb{} est alors décrite en explicitant d'une part les contraintes qu'elle doit permettre de surmonter, et d'autre part les solutions techniques mises en œuvre pour assurer une exploration interactive, intuitive et rapide des sorties de \simfeodal{}.
Cette plateforme, développée pour un cas spécifique mais conçue dans un cadre générique, rend ainsi l'exploration de ces données spatio-temporelles massives accessible à chacun.


Dans le \cref{chap:chap5} nous présentons le calibrage du modèle \simfeodal{} ainsi que les indicateurs de sortie du modèle issu de ce calibrage, et ce, grâce à l'utilisation de la plateforme \simedb{}.
Ces résultats sont analysés au regard de la grille d'évaluation fixée dans le \cref{chap:chap3}.
Une fois le modèle globalement calibré et satisfaisant du point de vue thématique, nous discutons de ce qui limite l'achèvement du calibrage, notamment du fait de l'incomplétude des sources empiriques disponibles et d'une compréhension insuffisante des effets des paramètres.
Nous menons alors une analyse visuelle de la sensibilité du modèle qui contribue à son évaluation générale et permet d'identifier des points potentiels d'amélioration de \simfeodal{}.
Le calibrage et l'analyse de sensibilité agissent ainsi comme des méthodes d'exploration du modèle, et permettent d'enrichir la compréhension des interactions entre ses différentes composantes.

Sous une forme \og pré-conclusive\fg{} et en réponse au \cref{chap:chap1}, le \cref{chap:chap6} est l'occasion d'effectuer un retour réflexif sur les spécificités du travail mené au cours de cette thèse.
En premier lieu, nous revenons sur les spécificités méthodologiques et techniques qu'imposent l'analyse de données de simulation.
En marquant la différence de ces dernières vis-à-vis de données plus classiques manipulées en sciences humaines et sociales, nous justifions et encourageons le recours à la visualisation pour explorer ces \og données intermédiaires\fg{} et favoriser, en cela, une posture abductive.
En second lieu, nous réalisons un retour d'expérience critique sur la posture de co-construction interdisciplinaire suivie lors de la construction et l'évaluation du modèle.
En analysant \textit{a posteriori} cette expérience collective, les implications de chacun et la trajectoire de \simfeodal{} au fil du temps, nous étayons l'importance d'une implication de chaque participant à chaque étape de la modélisation, comme cela a effectivement été le cas dans la construction et l'évaluation collective de ce modèle co-construit en interdisciplinarité.


\subsection*{Perspectives}

Ce manuscrit clôt un retour sur une expérience de plus de six années de travail, mais ouvre aussi la voie à de très nombreuses perspectives d'approfondissements et de prolongations.
La multiplicité des perspectives émergeant de ce travail relève selon nous de deux facteurs combinés.
D'une part, cela est dû à la diversité des types de propositions qui sont faites au cours de cette thèse :
	élaboration d'une approche conceptuelle de co-construction de modèle à base d'agents, réalisation du modèle \simfeodal{}, proposition méthodologique d'évaluation visuelle, etc.
D'autre part, le positionnement de la thèse à l'interface de plusieurs disciplines et la mobilisation de plusieurs approches parmi ces dernières renforce l'aspect foisonnant des perspectives. 

\paragraph{Poursuivre la modélisation de SimFeodal.}
Dans le \cref{chap:chap5}, nous avons identifié certaines limites à la poursuite du calibrage de \simfeodal{}, et donc à son amélioration.
L'analyse de sensibilité menée ensuite a néanmoins ouvert de nouvelles perspectives d'amélioration, en mettant notamment en évidence le rôle inattendu joué par certains paramètres.
Il serait fructueux, par la suite, de concentrer le travail sur ces paramètres, en particulier en poursuivant le calibrage du modèle à partir des paramètres ayant une influence nette sur un indicateur et peu d'effets secondaires sur les autres (\cref{sssec:sensib-params-seigneurs}). 

En nous fondant sur les récentes avancées en simulation informatique distribuée, on pourrait chercher à prolonger l'exploration de \simfeodal{}.
Pour cela, il est possible de faire appel soit à différentes méthodes d'optimisation en identifiant des \og fonctions objectifs\fg{}, soit à des méthodes d'exploration automatique du paysage des sorties du modèle.
Parmi celles-ci, il serait intéressant de tester l'influence réelle de la situation initiale de \simfeodal{}, qui est générée aléatoirement suivant les valeurs de paramètres d'\textit{input}.
C'est un travail qui a été fait sur des modèles KISS \autocite{raimbault_space_2019} et qui nous semble intéressant à mettre en oeuvre en vue, par exemple, de tester l'hypothèse d'isotropie qui est au coeur de la conception de \simfeodal{}.
En testant différentes situations initiales de manière plus systématique, on pourrait alors chercher à tester l'effet de configurations initiales plus variées, par exemple dans le cas de la modélisation des dynamiques spatiales que l'on analyserait sur d'autres régions d'étude.

Toujours dans le domaine de la simulation, sur le plan théorique, \simfeodal{} gagnerait sans doute à être \og modularisé\fg{}.
Cela pourrait se faire en transformant le modèle pour le doter d'une structure plus générique, sur laquelle on pourrait dès lors connecter (ou non) différents \og modules\fg{} correspondant aux spécificités du modèle.
En s'engageant dans cette dynamique de multi-modélisation \autocite{cottineau_chapter_2019}, il serait par exemple possible de tester des situations avec ou sans églises paroissiales, en ajoutant d'autres types d'attracteurs dotés de mécanismes propres, en modifiant les mesures de satisfaction, etc.
La nature très paramétrique de \simfeodal{} permet déjà d'effectuer certains de ces tests, mais des changements plus importants des mécanismes risqueraient de rompre la rétro-compatibilité du modèle.
Avec un modèle modulaire, toutes les variantes qui ont été testées puis abandonnées au cours de la construction et du paramétrage du modèle pourraient véritablement être confrontées \textit{ceteris paribus}.

Cette réalisation théorique et technique permettrait par ailleurs de faciliter la conduite de nouveaux cas d'études thématiques et empiriques.
Ainsi, du point de vue de la modélisation en archéologie et en histoire, on gagnerait nécessairement en connaissances sur les processus spatiaux modélisés en menant une \og validation croisée\fg{} thématique du modèle, comme proposé dans le chapitre précédent (\cref{subsec:perspectives-validation}, p.~\pageref{par:validation-croisee}).
Il s'agirait ainsi de tester la généricité du modèle en l'adaptant à d'autres cas d'études, par exemple à d'autres régions, où l'on retrouve les mêmes dynamiques spatiales que celles observées en Touraine.
À l'inverse, tester le modèle sur des régions où les facteurs semblent comparables mais où les structures spatiales résultantes sont très différentes -- c'est pas exemple le cas du Quercy où il n'y a eu qu'une faible polarisation de l'habitat -- permettrait aussi de progresser dans la compréhension de la conjonction de facteurs nécessaires et suffisants pour qu'apparaissent la polarisation, la hiérarchisation et la fixation du peuplement.
Avec la possibilité méthodologique d'activer et de désactiver les \og modules\fg{} de \simfeodal{}, il serait enfin possible de systématiser la simulation de ces cas d'études spécifiques, et ainsi d'être en mesure de généraliser plus fortement nos hypothèses thématiques.

Dans l'ensemble, toutes ces pistes héritées de la modélisation en sciences sociales et en sciences de la complexité permettraient, pour la géographie, d'éclairer les conditions d'émergence des phénomènes de polarisation et de hiérarchisation, constatés dans une grande diversité de systèmes géographiques, tant en termes d'époques que d'espaces.
En développant et en élargissant l'analyse de ce cas d'étude, on contribuerait ainsi à discuter les différentes théories qui proposent d'expliquer l'organisation spatiale des systèmes sociaux (auto-organisation \autocite{saint1989villes}, attachement préférentiel \autocite{albert_statistical_2002}, théorie évolutive des villes \autocite{pumain_pour_1997}, etc.).

\paragraph{Développement et mise à l'épreuve de l'évaluation visuelle.}
Dans le \cref{chap:chap3}, nous avons proposé une méthode d'évaluation des modèles basée sur l'analyse visuelle d'indicateurs de sortie de simulation nombreux et variés.
Dans le cas de \simfeodal{}, l'usage de cette méthode a été utile et fructueux.
Toutefois, il nous semble que l'on pourrait aller plus loin dans son usage.

Une première piste, largement inspirée par la géomatique et l'analyse spatiale, serait de chercher à mieux rendre compte, visuellement, des configurations spatiales produites par le modèle.
Dans le \cref{chap:chap6} (\cref{subsec:genericite-donnees-simul}, p.~\pageref{par:specificites-donnees-simul}), nous présentions ainsi les difficultés de l'exécution de \og résumés statistiques\fg{} (sur la dimension réplicative notamment) des données géographiques issues d'un espace théorique et principalement aléatoire.
Pour améliorer la qualité et la vitesse de l'évaluation des sorties d'une version du modèle, nous sommes convaincus que l'utilisation de plus nombreux indicateurs issus de l'analyse spatiale et des géostatistique serait plus efficace que la démultiplication des représentations spatiales provenant de chaque simulation.
À l'aide d'indicateurs de dispersion, de méthodes de classification spatiale de l'espace, ou encore de résumés de lissages cartographiques, on serait ainsi en mesure de \og dé-spatialiser\fg{} des indicateurs relatifs à l'espace, et dès lors de les agréger à travers différentes temporalités, réplications ou expériences, pour en comparer par exemple l'évolution moyenne au cours du temps simulé.

Une autre limite, spécifique à la réalisation technique de la plateforme d'exploration des données \simedb{}, est la difficulté croissante d'interrogation rapide des données à mesure que celles-ci augmentent.
Comme l'après-propos du \cref{chap:chap4} l'indique, nous n'avons finalement pas été en mesure de stocker et interroger les données individuelles des foyers paysans lors des dernières phases d'exploration du modèle.
Avec la production de 20 millions de lignes de données pour chaque expérience, sur une quinzaine d'expériences, on atteignait alors les limites techniques de l'environnement utilisé, limites difficilement dépassables sans changer d'infrastructure informatique matérielle.
En reprenant la typologie du \cref{chap:chap6} (\cref{tab:donnees-intermediaires}, p.~\pageref{tab:donnees-intermediaires}), on approchait ainsi assez largement d'une volumétrie de big data et d'exigences de calcul intensif.
Le champ scientifique de l'informatique graphique nous semble proposer deux approches opposées pour résoudre ce problème d'interrogation rapide de jeux de données toujours plus massifs, sans faire pour autant appel à des méthodes d'optimisation plus coûteuses matériellement \autocite{amirpour_amraii_human-data_2018}.
Il s'agit, pour l'interrogation des données, d'avoir recours soit à l'approximation, soit à des requêtes incrémentales \autocite[28--33]{amirpour_amraii_human-data_2018}.

L'approximation consiste à effectuer une requête rapide, quitte à ne renvoyer qu'une approximation heuristique du résultat.
Cette possibilité dépend avant tout du système de gestion de base de données (SGBD) qui organise les données. Ce dernier doit être capable de produire une approximation aussi exacte que possible pour retourner son résultat au regard du temps maximal qu'on lui donne.
C'est un enjeu majeur de la recherche en bases de données pour traiter des corpus de plus en plus importants de manière interactive.
Selon les mots de \textcite[7]{fekete_visual_2013} :
	\og Providing the mechanisms for exploration in databases and analysis systems will benefit to all the situations when users are willing to trade accuracy for time, an important issue since time is becoming one of our most important resources\fg{}.
De nouvelles solutions logicielles permettant de telles requêtes sont régulièrement proposées \autocite[par exemple EntropyDB, de ][]{orr_entropydb_2019}, mais la relative jeunesse de ce champ le rend encore instable.
Pour exemple, la solution BlinkDB \autocite{agarwal_blinkdb_2013} présentée dans le \cref{chap:chap4} a depuis été abandonnée, notamment au profit de EntropyDB.

La faculté de mener des requêtes incrémentales s'inscrit dans la recherche en informatique, autour de l'idée de \og \textit{progressive visual analytics}\fg{} \autocite{6876049}, de \og \textit{progressive analytics}\fg{} \autocite{fekete_progressive_2016}, ou encore de \og \textit{progressive data science}\fg{} \autocite{turkay_progressive_2018}.
Plutôt que de demander à une base de données de renvoyer une unique approximation d'un résultat, il s'agit de récupérer régulièrement des résultats de plus en plus précis, correspondant à des approximations successives de plus en plus fines.
Contrairement aux méthodes d'approximation, ces approches \og progressives\fg{} ont l'avantage d'être déterministes (la dernière approximation correspond en fait au résultat \og objectif\fg{} de la requête) et donc de permettre des comparaisons très rapides entre résultats de simulation, tout en permettant de renvoyer des résultats précis au prix d'une attente un peu plus élevée.
La principale difficulté d'intégration de telles méthodes à une plateforme existante telle que \simedb{} est qu'il faut changer non seulement le SGBD, mais aussi le mode de requête afin que l'affichage des indicateurs de sortie soit actualisé régulièrement pour prendre en compte les nouvelles approximations.
Cette piste demande donc d'importants développements informatiques, mais nous semble extrêmement stimulante comme possibilité d'interrogation de données très massives.

Une autre piste d'amélioration de la plateforme, liée cette fois à l'analyse de données exploratoires, porte sur la collecte de retours utilisateurs automatisés en vue de fouille automatisée de données. Cette idée est présente dans notre travail depuis le début de la thèse, mais n'a pu être menée à bien par faute de temps.
L'enjeu est de récupérer les \og notations\fg{} des indicateurs de chaque simulation (décrits dans la \cref{sssec:modes-interaction}, p.~\pageref{par:noter-simul}) et d'utiliser ces données comme une source d'analyse.
Par exemple, avec des analyses factorielles portant sur les indicateurs de sortie, les expériences qu'ils décrivent et les notes données par chacun des participants, il devrait être possible d'identifier de manière quantitative les expériences les plus satisfaisantes, mais aussi les indicateurs qui sont les moins utiles, ou encore d'identifier des expériences qui auraient été peu évaluées visuellement mais au profil proche d'expériences notées positivement.

De telles approches de \og fouille de données\fg{} (\textit{data mining}), voire d'apprentissage automatique, constitueraient en outre un premier pas vers une évaluation de la plateforme \simedb{} et des choix effectués dans le cadre de sa conception, ce qui constitue une perspective importante de ce travail.
L'évaluation des méthodes visuelles est en effet au cœur de nombreuses pratiques de la communauté d'informatique graphique et nous semble indispensable pour juger de la validité de l'approche mise en œuvre dans ce travail de thèse.
Dans le cas de \simedb{}, une telle analyse est difficile car les utilisateurs de la plateforme sont captifs, dans le sens où il n'ont pas d'alternative pour évaluer les sorties de \simfeodal{}.
\simedb{} ne s'adresse pas non plus à un grand public, et le recrutement de \og testeurs\fg{} externes, pour mesurer l'effectivité de notre plateforme à réaliser les tâches d'évaluation du modèle, serait aussi vain.
L'évaluation de notre proposition de méthode d'évaluation visuelle est donc un chantier de recherche important, sans doute difficilement réalisable en tant que tel.
Pour aller dans ce sens, cependant, l'application de méthodes d'évaluation visuelle à d'autres modèles serait sans doute un premier pas non négligeable et qui pourrait de plus être mené dans un délai raisonnable, par exemple dans le cadre de projets de modélisation interdisciplinaire déjà en cours.

\paragraph{Généraliser l'approche de co-construction.}
Ces réflexions conclusives sur l'évaluation de la méthode d'évaluation visuelle proposée dans ce travail nous semblent centrales.
On pourrait même les généraliser à l'ensemble des propositions de cette thèse :
	la réalisation d'évaluations -- qu'elles portent sur un modèle, sur une approche de co-construction, sur une démarche d'évaluation visuelle ou encore sur un outil interactif d'exploration -- permet systématiquement de gagner en connaissance sur ce qui est évalué, et ainsi de l'améliorer.
Cette thèse, comme indiqué dans l'Introduction générale, constitue avant tout une proposition méthodologique.
À ce titre, la meilleure évaluation globale consiste en la reproduction des approches qui y sont promues sur des cas d'études variés.
Cela est d'autant plus vrai concernant l'évaluation de l'approche de co-construction interdisciplinaire de modèle.
En co-construisant de nouveaux modèles, potentiellement avec d'autres disciplines des sciences humaines et sociales, cette approche horizontale fondée sur l'accompagnement à la modélisation ne peut qu'être étayée et éprouvée.
De la même manière que la formalisation de démarche de modélisation d'accompagnement, ComMod \autocite{commod_modelisation_2005} résulte avant tout d'une somme d'expériences partagées et synthétisées par un collectif pluriel de chercheurs impliqués en recherche-action, il nous semble que la multiplication des expériences de co-construction interdisciplinaire de modèles en sciences humaines et sociales serait le moyen privilégié d'étoffer ou de réviser les propositions de ce travail.