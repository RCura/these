% !TEX root = ../These_Robin_Master.tex
\pagestyle{empty}
\setstretch{1}
{\large\textbf{Modéliser des systèmes de peuplement en interdisciplinarité.\\ Co-construction et exploration visuelle d'un modèle de simulation.}}

\textbf{Résumé}

\vspace*{-0.5em}
Cette thèse vise à expérimenter une démarche collective et interdisciplinaire d'analyse de dynamiques spatiales sur le temps long, en se fondant sur la co-construction et l'exploration visuelle d'un modèle de simulation des transitions des systèmes de peuplements.
Il s'agit de simuler les transformations du système de peuplement Nord-Ouest européen entre 800 et 1100, période de transition d'un système majoritairement dispersé à un système hiérarchisé et polarisé.
L'objectif est de modéliser et d'analyser les interactions à différentes échelles qui ont provoqué ces transformations et engendré les dynamiques de polarisation, de hiérarchisation et de fixation de l'habitat.\\
Pour répondre à cet objectif, la thèse déploie une démarche méthodologique de co-construction d'un modèle de simulation avec des modélisateurs et des experts thématiciens archéologues, historiens et géographes.
Cette démarche permet d'accompagner les thématiciens dans la modélisation conceptuelle et informatique du système étudié, via le développement du modèle à base d'agents \simfeodal{}.
Le modèle est collectivement évalué, paramétré et exploré grâce à la mise en oeuvre d'une méthode d'évaluation visuelle.
On peut alors tester les hypothèses sous-jacentes aux processus mobilisés pour décrire les dynamiques spatiales modélisées en comparant systématiquement les sorties du modèle.
La comparaison est rendue possible par l'utilisation d'un environnement d'exploration, la plateforme \simedb{}.
Cet outil interactif conçu spécifiquement pour cet usage permet l'exploration des données spatio-temporelles massives issues du modèle, et facilite ainsi l'analyse collective et interdisciplinaire des hypothèses thématiques.

\textbf{Mots-clés :} 
modélisation à base d'agents; visualisation; systèmes de peuplement; temps long; interdisciplinarité; SimFeodal; SimEDB.

\vspace{1.2em}
{\large\textbf{Interdisciplinary modelling of settlement systems through the collective co-construction and visual analysis of a simulation model.}}

\textbf{Abstract}

\vspace*{-0.5em}
This thesis experiments a collaborative and interdisciplinary approach for the analysis of long-term spatial dynamics, based on the collective construction and visual exploration of a model which simulates the transitions of a settlement system.
The case study is the Northwest European settlement system which between 800 and 1100 transitioned from a mostly dispersed settlement system to a hierarchical and polarized one: the objective of this research is to model and analyse the multiscalar interactions that resulted in these transformations and generated the dynamics of polarisation, hierarchisation and settlement fixation.\\
To this end, this thesis proposes a methodological approach based on the co-construction of a simulation model involving experts from various disciplines (archaeology, history and geography).
This approach proposes to support the domain experts for conceptualizing and modelling the system under study, through the development of \simfeodal{}, an agent-based model.
This model is collectively evaluated, parameterized and explored through methods of visual evaluation.
The methods rely on the systematic comparison of the model's outputs to test the thematic hypothesis underlying the choice and selection of processes which simulates the settlement system's spatial dynamics.
The \simedb{} platform, specifically designed for this use, provides an interactive tool to explore the massive spatio-temporal datasets of the model, and thus facilitates the collective and interdisciplinary analysis of the thematic hypotheses.

\textbf{Keywords :}
agent-based modeling, visualization, settlement systems, long term, interdisciplinarity, SimFeodal, SimEDB.