\usepackage[utf8]{inputenc}
\usepackage{amsmath}
\usepackage{amsfonts}
\usepackage{amssymb}
\usepackage{graphicx}
\usepackage[left=1.5cm, right=5cm, bottom = 1.5cm, top = 1.5cm]{geometry}
\setlength{\marginparwidth}{4.5cm}
\usepackage{marginnote}

\usepackage{fontspec}
\usepackage[french]{babel}
\usepackage[babel=true]{csquotes}

\usepackage{soul}
\usepackage{mdframed}

\usepackage{xargs} % Use more than one optional parameter in a new commands
\usepackage[dvipsnames]{xcolor}

\usepackage{float}
\usepackage{caption}

\usepackage{fancyhdr}
\pagestyle{fancy}

\renewcommand{\headrulewidth}{0.5pt}
\renewcommand{\footrulewidth}{0pt}
\fancyhead[L]{Chapitre \thechapter}
\fancyhead[C]{}
\fancyhead[R]{\rightmark}


\renewcommand{\thempfootnote}{\arabic{footnote}}%

\usepackage{enumitem}

\usepackage[hyperfootnotes=false]{hyperref}
\usepackage[noabbrev, nameinlink]{cleveref}

\usepackage{longtable}
\usepackage{array}

\usepackage[style=authoryear-comp,
hyperref,
backend=bibtex,
isbn=false,
doi=true,
url=true,
date=year]{biblatex}

\usepackage[thinlines]{easytable}
\usepackage{makecell}
\usepackage{diagbox}

\renewcommand\theadalign{bc}
\renewcommand\theadfont{\bfseries}
\renewcommand\theadgape{\Gape[4pt]}
\renewcommand\cellgape{\Gape[4pt]}

\usepackage[french]{minitoc}
\dominitoc


\usepackage{sparklines}

% ------------------------------------- %
% ---------- COMMANDES PERSO ---------- %
% ------------------------------------- %
% ######### TABLES AVEC FIGURES #########
\newcolumntype{M}[1]{>{\centering\arraybackslash}m{#1}}
\newcolumntype{N}{@{}m{0pt}@{}}
\usepackage{tikz}
\usetikzlibrary{calc,shapes,arrows}
\newcommand{\tikzmark}[1]{%
	\tikz[overlay,remember picture] \node (#1) {};}

%%% TABLES
\usepackage{verbatim}
\usepackage{makecell}
\usepackage{multirow}
%\setcounter{secnumdepth}{4}

\crefname{paragraph}{paragraphe}{paragraphes}
\Crefname{paragraph}{Paragraphe}{Paragraphes}

\makeatletter
\DeclareRobustCommand{\cnameref}[1]{%
%	\namecref{#1}
	\nameref{#1}%
}%
\DeclareRobustCommand{\Cnameref}[1]{%
	\nameCref{#1} \nameref{#1}%
}

% ###### CREATION D'ENCADRES ###### %

% Encadré pris dans la thèse de Seb Rey
\usepackage[most]{tcolorbox}
\newtcbtheorem[number within=chapter]{encadre}{Encadré}{   
	outer arc=0pt,
	arc=0pt,
	breakable,
	enhanced,
	colback=white,
	colframe=black,
	colbacktitle=white,
	titlerule=0pt,
	fonttitle=\normalcolor\itshape}{enc}

\crefformat{tcb@cnt@encadre}{encadré~#2#1#3}
\Crefformat{tcb@cnt@encadre}{Encadré~#2#1#3}

% ###### COMMENTAIRES ###### %


% Surlignement texte (1) + commentaire en marge (2)

\DeclareRobustCommand{\hlorange}[1]{{\sethlcolor{Dandelion}\hl{#1}}}
\DeclareRobustCommand{\hlcyan}[1]{{\sethlcolor{Cyan}\hl{#1}}}

\newcommandx{\toChange}[2]{%
	\colorbox{Dandelion}{#1}\marginnote{\small\hlorange{#2}}%
}

\newcommandx{\Lena}[2]{%
	\colorbox{Cyan}{#1}\marginnote{\small\hlcyan{Lena:\\ #2}}%
}

% Highlight
\newcommandx{\fixref}[1]{\hl{#1}}

% ToDoBox
\newcommandx{\todobox}[2][1=]{%
	~\\	\colorbox{pink}{\parbox{0.9\textwidth}{%
		\vskip5pt
		\leftskip5pt\rightskip5pt
		#2
		\vskip5pt
	}~\\
}
}

% ######### TITRE #########
\usepackage{titling}
\title{Accompagner la modélisation des systèmes de peuplement par l’exploration interactive de données spatio-temporelles}
\author{Robin Cura}
\date{\vspace{-5ex}}
\makeatletter
\newcommand*{\toccontents}{\@starttoc{toc}}
\makeatother


% ######### COULEURS POUR REPRISE #########
%\setmainfont[Color=red]{Linux Libertine O}
%\setsansfont[Color=red]{Linux Biolinum O}
%\setmonofont[Color=red]{Courier}
% On linux
\newfontfamily\blueroman[Color=blue]{Linux Libertine O}
\newfontfamily\redroman[Color=red]{Linux Libertine O}
% On windows
%\newfontfamily\blueroman[Color=blue]{Linux Libertine}
%\newfontfamily\redroman[Color=red]{Linux Libertine} 


% ######### PLAN DETAILLE #########
% Make a TOC without line number when calling \tableofcontents
%\let\Contentsline\contentsline
%\renewcommand\contentsline[3]{\Contentsline{#1}{#2}{}}
