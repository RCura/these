\usepackage[utf8]{inputenc}
\usepackage{amsmath}
\usepackage{amsfonts}
\usepackage{amssymb}
\usepackage{graphicx}
\usepackage[left=2.00cm, right=5.00cm, bottom = 2.00cm, top = 2.00cm]{geometry}
\setlength{\marginparwidth}{4cm}

\usepackage[T1]{fontenc}

\usepackage[francais]{babel}

\usepackage{mdframed}

\usepackage{xargs} % Use more than one optional parameter in a new commands
\usepackage[dvipsnames]{xcolor}
\usepackage[colorinlistoftodos,prependcaption,textsize=tiny]{todonotes}

\usepackage{graphicx}

\usepackage{tikz}
\usetikzlibrary{shapes.misc}

\usepackage{newfloat}
\usepackage{caption}

\usepackage{fancyhdr}
\pagestyle{fancy}

\renewcommand{\headrulewidth}{0.5pt}
\renewcommand{\footrulewidth}{0pt}
\fancyhead[L]{Chapitre \thechapter}
\fancyhead[C]{}
\fancyhead[R]{\rightmark}


\usepackage{enumitem}

\usepackage{hyperref}
\usepackage{cleveref}








% ------------------------------------- %
% ---------- COMMANDES PERSO ---------- %
% ------------------------------------- %

% ###### CREATION D'ENCADRES ###### %

\DeclareFloatingEnvironment[fileext=frm,
placement={htb},
listname=Liste des encadrés,
within=chapter,
name=BLOB
]{myencadre}


\newcounter{encadre}[chapter]


\newenvironment{encadre}[1]{%
	\stepcounter{encadre}
	\mdfsetup{%
		frametitle={
			\tikz[baseline=(current bounding box.east),outer sep=0pt]
			\node[anchor=east,rounded rectangle,
			rounded rectangle west arc = 0pt,
			fill=black!20]
			{\strut Encadré \thechapter.\arabic{encadre}: \captionlistentry{#1}{#1}};},
		innertopmargin=10pt,linecolor=black!20,
		linewidth=2pt,topline=true,
		innerrightmargin=10pt,innerleftmargin=20pt,
		frametitleaboveskip=\dimexpr-\ht\strutbox\relax
	}
	\begin{myencadre}[!h]\begin{mdframed}[]\relax%
		}{\end{mdframed}\end{myencadre}}

\crefname{myencadre}{encadré}{encadrés}
\Crefname{myencadre}{Encadré}{Encadrés}

% Exemple d'encadré
% \begin{encadre}{Test\label{enc:bar}}
% ATTENTION : Le label ne peut être mis que là
% \end{encadre}
% Appeler en fin de manuscrit avec
%  \listofmyencadres
% ------------------------------------- %

% ###### TODO INLINE ###### %
\newcommandx{\unsure}[2][1=]{\todo[linecolor=red,backgroundcolor=red!25,bordercolor=red,#1]{#2}}

\newcommandx{\unsurec}[2][1=]{%
	\colorbox{red!25}{#2}\todo[linecolor=red,backgroundcolor=red!25,bordercolor=red,#1]%
}

\newcommandx{\change}[2][1=]{\todo[linecolor=blue,backgroundcolor=blue!25,bordercolor=blue,#1]{#2}}
\newcommandx{\changec}[2][1=]{%
	\colorbox{blue!25}{#2}\todo[linecolor=blue,backgroundcolor=blue!25,bordercolor=blue,#1]%
}

\newcommandx{\info}[2][1=]{\todo[linecolor=OliveGreen,backgroundcolor=OliveGreen!25,bordercolor=OliveGreen,#1]{#2}}
\newcommandx{\infoc}[2][1=]{%
	\colorbox{OliveGreen!25}{#2}\todo[linecolor=OliveGreen,backgroundcolor=OliveGreen!25,bordercolor=OliveGreen,#1%
}


\newcommandx{\improvement}[2][1=]{\todo[linecolor=Plum,backgroundcolor=Plum!25,bordercolor=Plum,#1]{#2}}
\newcommandx{\improvementc}[2][1=]{%
	\colorbox{Plum!25}{#2}\todo[linecolor=Plum,backgroundcolor=Plum!25,bordercolor=Plum,#1]%
}

% ###### FIXME INLINE ###### %
\newcommandx{\fixref}[1]{#1}