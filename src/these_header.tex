%\usepackage[utf8]{inputenc}
\usepackage{amsmath}
\usepackage{amsfonts}
\usepackage{amssymb}
\usepackage{graphicx}
\usepackage[left=1.5cm, right=5cm, bottom = 1cm, top = 2cm, headheight=16pt,
headsep = 0.5cm, foot = 16pt, footskip = 0.5cm]{geometry}
\setlength{\marginparwidth}{4.5cm}
\usepackage{marginnote}

\usepackage{fontspec}
\setmainfont{Charis SIL}
\usepackage{setspace}
\setstretch{1.1}
\usepackage[french]{babel}
\usepackage[babel=true]{csquotes}

\usepackage{soul}
\usepackage{mdframed}

\usepackage{xargs} % Use more than one optional parameter in a new commands
\usepackage[dvipsnames,table]{xcolor}


\usepackage{float}
\usepackage{caption}
\captionsetup[figure]{belowskip=-0.5cm}
\captionsetup[lstlisting]{belowskip=-0.25cm}
%\captionsetup[table]{position=above, aboveskip=0pt, belowskip=10pt}
\usepackage{rotating}
\usepackage{rotfloat}

\usepackage{fancyhdr}
\pagestyle{fancy}

\renewcommand{\headrulewidth}{0.5pt}
\renewcommand{\footrulewidth}{0pt}
\fancyhead[L]{Chapitre \thechapter}
\fancyhead[C]{}
\fancyhead[R]{\rightmark}

\usepackage{epigraph}

\newcounter{savefootnote}
\renewcommand{\thempfootnote}{\arabic{footnote}}%

\usepackage{enumitem}
\setlist{  
	listparindent=\parindent,
	parsep=0pt,
}

\usepackage[hyperfootnotes=false]{hyperref}
\hypersetup{pdftitle={Robin Cura - Thèse}}
\usepackage[noabbrev, nameinlink]{cleveref}

\usepackage{longtable}
\usepackage{array}

\usepackage{fontawesome}
\usepackage{wrapfig}
\usepackage[style=authoryear-comp,
hyperref,
backend=bibtex,
isbn=false,
doi=true,
url=true,
date=year,
sortcites=false,  % Ne pas changer l'ordre des citations multiples
sorting=nyt % Citations dans biblio : NameYearTitle
]{biblatex}


%%%% Styles chapitres %%%%
\usepackage[Bjornstrup]{fncychap}
\ChTitleVar{\raggedright\LARGE\sffamily\bfseries}


%\usepackage{natbib}

\usepackage[thinlines]{easytable}
\usepackage{makecell}
\usepackage{diagbox}
%\usepackage{colortbl}      
\usepackage{hhline}     

\renewcommand\theadalign{bc}
\renewcommand\theadfont{\bfseries}
\renewcommand\theadgape{\Gape[4pt]}
\renewcommand\cellgape{\Gape[4pt]}

\usepackage[french, tight]{minitoc}
\dominitoc

\usepackage{listings} % Code
\lstset{
	basicstyle=\ttfamily,
	frame=single
}
%\usepackage{minted}
\renewcommand\lstlistingname{Code}
\renewcommand\lstlistlistingname{Code}
\crefformat{lstlisting}{code~#2#1#3}
\Crefformat{lstlisting}{Code~#2#1#3}

\usepackage[lofdepth,lotdepth]{subfig}
\usepackage{sparklines}

% ------------------------------------- %
% ---------- COMMANDES PERSO ---------- %
% ------------------------------------- %
% ######### TABLES AVEC FIGURES #########
\newcolumntype{M}[1]{>{\centering\arraybackslash}m{#1}}
\newcolumntype{N}{@{}m{0pt}@{}}
\usepackage{tikz}
\usetikzlibrary{calc,shapes,arrows}
\newcommand{\tikzmark}[1]{%
	\tikz[overlay,remember picture] \node (#1) {};}
% ############# SCHEMAS TIKZ ###########
\usepackage{tikz}
\usetikzlibrary{shapes,arrows.meta, shapes.geometric, positioning, patterns}
\usepackage{dashbox}

%%% TABLES
\usepackage{verbatim}
\usepackage{makecell}
\usepackage{multirow}

\usepackage{upgreek}

\setcounter{secnumdepth}{6}
\crefname{paragraph}{paragraphe}{paragraphes}
\Crefname{paragraph}{Paragraphe}{Paragraphes}

\makeatletter
\DeclareRobustCommand{\cnameref}[1]{%
%	\namecref{#1}
	\nameref{#1}%
}%
\DeclareRobustCommand{\Cnameref}[1]{%
	\nameCref{#1} \nameref{#1}%
}

% ###### CREATION D'ENCADRES ###### %

% Encadré pris dans la thèse de Seb Rey
\usepackage[most]{tcolorbox}
\newtcbtheorem[number within=chapter]{encadre}{Encadré}{
	outer arc=0pt,
	arc=0pt,
	breakable,
	enhanced,
	colback=white,
	colframe=black,
	colbacktitle=white,
	titlerule=0pt,
	fonttitle=\normalcolor\itshape}{enc}

% A utiliser avec \usepackage{cleveref}
\crefformat{tcb@cnt@encadre}{encadré~#2#1#3}
\Crefformat{tcb@cnt@encadre}{Encadré~#2#1#3}

%%% UTILISATION %%%
% \begin{encadre}{Titre de l'encadré}{label-de-lencadre}
% Contenu de l'encadré
%\end{encadre}
% Référence interne dans le texte :\cref{enc:label-de-lencadre}



% ###### COMMENTAIRES ###### %


% Surlignement texte (1) + commentaire en marge (2)

\DeclareRobustCommand{\hlorange}[1]{{\sethlcolor{Dandelion}\hl{#1}}}
\DeclareRobustCommand{\hlcyan}[1]{{\sethlcolor{Cyan}\hl{#1}}}

\newcommandx{\toChange}[2]{%
	\colorbox{Dandelion}{#1}\marginnote{\small\hlorange{#2}}%
}

\newcommandx{\Lena}[2]{%
	\colorbox{Cyan}{#1}\marginnote{\small\hlcyan{Lena:\\ #2}}%
}

% Highlight
\newcommandx{\fixref}[1]{\hl{#1}}

% ToDoBox
\newcommandx{\todobox}[2][1=]{%
	~\\	\colorbox{pink}{\parbox{0.9\textwidth}{%
		\vskip5pt
		\leftskip5pt\rightskip5pt
		#2
		\vskip5pt
	}~\\
}
}

% ######### TITRE #########
\usepackage{titling}
\title{Accompagner la modélisation des systèmes de peuplement par l’exploration interactive de données spatio-temporelles}
\author{Robin Cura}
\date{\vspace{-5ex}}
\makeatletter
\newcommand*{\toccontents}{\@starttoc{toc}}
\makeatother




\newskip\bigskipamount   \bigskipamount =20pt plus 4pt minus 4pt
\setlength{\parskip}{0.75em}
\setlength{\parindent}{2em}
\setlength{\footnotesep}{.75\baselineskip}

\usepackage{titlesec} % Pose problème avec les headers
% Corrigé avec ce code, à mettre pour chaque section
%\let\orisectionmark\sectionmark
%\renewcommand\sectionmark[1]{}%
%\section[Titre TOC]{Titre dans texte}
%\orisectionmark{Titre header}
%\let\sectionmark\orisectionmark

\titlespacing\section{0pt}{16pt plus 4pt minus 2pt}{0pt plus 2pt minus 2pt}
\titlespacing\subsection{0pt}{11pt plus 4pt minus 2pt}{0pt plus 2pt minus 2pt}
\titlespacing\subsubsection{0pt}{11pt plus 4pt minus 2pt}{0pt plus 2pt minus 2pt}
\titlespacing\paragraph{2em}{6pt plus 2pt minus 2pt}{8pt plus 0pt minus 0pt}



% ######### PLAN DETAILLE #########
% Make a TOC without line number when calling \tableofcontents
%\let\Contentsline\contentsline
%\renewcommand\contentsline[3]{\Contentsline{#1}{#2}{}}
\usepackage{pbox}
\usepackage{array}% http://ctan.org/pkg/array



% ###### AVANT-PROPOS CHAP2 ######
\usepackage{multicol}


% ######## Plusieurs ndbp renvoyant à la même ######
\usepackage[bottom]{footmisc}
% # Et on met un asterisque à la place
\newcommand{\astfootnote}[1]{
	\let\oldthefootnote=\thefootnote
	\setcounter{footnote}{0}
	\renewcommand{\thefootnote}{\fnsymbol{footnote}}
	\hspace*{-.45cm}\footnote{#1}\unskip
	\let\thefootnote=\oldthefootnote
}

% ############ CITATIONS #############
% Citer juste le prénom + nom d'une référence
\newrobustcmd*{\citeauteur}{\AtNextCite{\DeclareNameAlias{labelname}{first-last}}\citeauthor}
% \citeauteur{ref}