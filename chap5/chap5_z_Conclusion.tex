\clearpage
\section*{Conclusion}
\addcontentsline{toc}{section}{\protect\numberline{}Conclusion}

\hl{A reprendre !}\\
% TODO : Reprendre conclusion
Au terme de la construction de la plate-forme d'exploration, nous disposons donc d'une application, SimEDB, conçue et développée spécifiquement pour les problématiques propres à l'exploration des données de SimFeodal.

Elle s'inscrit dans les méthodes des Interactions Homme-Machine, ou même dans ce que certains nomment désormais les \og Interactions Homme-Données\fg{} (\og \textit{Human-Data Interaction}\fg{}, \cite{elmqvist_embodied_2011,mortier_human-data_2014}) et s'efforce de suivre les préceptes identifiés dans ce champs (\cite[167-170]{amirpour_amraii_human-data_2018} par exemple).

Le développement a été fortement guidé par les contraintes et besoins identifiées, aussi bien en terme d'approches méthodologiques que de choix technologiques.
SimEDB est donc un outil \textit{ad-hoc}, toutefois pensé de manière modulaire.
Tous les composants logiciels de SimEDB sont indépendants et communiquent de manière standardisée, ouvrant la voie à leur remplacement ou \og interchangeabilité\fg{} : l'architecture logicielle et les choix techologiques le permettent.
La plate-forme SimEDB est donc intrinsèquement pensée comme une réponse à des besoins spécifiques, mais cette réponse a été conçue comme générique et en mesure d'être adaptée aisément à d'autres types de données et/ou sorties de modèles de simulation.

Plus généralement, l'ensemble de ce chapitre montre une démarche similaire, pensée pour répondre à des besoins spécifiques avec des solutions génériques et généralisables.
Le passage, depuis une succession de rapports jusqu'à une application d'exploration de ces rapports, ou encore les différents éléments relatifs au choix d'un système de gestion de base de données ou au dessin d'un modèle conceptuel de données s'inscrivent en effet dans cette même démarche qui s'ancre profondément dans une logique de recherche reproductible, aussi bien d'un point de vue technique que de celui du concept et de la méthodologie.