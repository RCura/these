%%%%%%%%%%%%%%%%%%%%%%%%%%%%%%%%%%%%%%%%%%%%
% VERSION SANS LES INDICATEURS DE CONTEXTE %
%%%%%%%%%%%%%%%%%%%%%%%%%%%%%%%%%%%%%%%%%%%%%
\begin{table}[H]
	\captionsetup{singlelinecheck=off}
	\centering
	\small
	\resizebox{\textwidth}{!}{%
	{\renewcommand{\arraystretch}{1.3}%
		\begin{tabular}{|p{3cm}|p{2.2cm}|p{1.5cm}|p{1.5cm}|p{1.5cm}|p{1.5cm}|p{1.5cm}|}
			\hline
			\textbf{Indicateur} & \textbf{Valeur} \textbf{attendue}%\footnotemark
			 & \textbf{Moyenne} & \textbf{Médiane} & \textbf{Q1} & \textbf{Q3} & \textbf{Écart-type} \\ \hline
			\textit{Agrégats} & \textit{200} & 249 & 248 & 244 & 253 & 10.45 \\ \hline
			\textit{Gros châteaux} & \textit{10} & 15 & 15 & 13 & 17 & 2.87 \\ \hline
			\textit{Églises paroissiales} & \textit{300} & 348 & 348 & 338 & 359 & 12.96 \\ \hline
			\textit{Distance moyenne entre églises} & \textit{3 000 m} & 1 459 m & 1 456 m & 1 391 m & 1 537 m & 97 m \\ \hline
			\textit{Part de foyers paysans isolés} & \textit{20 \%} & 30 \% & 30 \% & 30 \% & 30 \% & 0.8 \% \\ \hline
			\textit{Augmentation de la charge fiscale des foyers paysans} & \textit{×3} & ×2.4 & ×2.4 & ×2.4 & ×2.5 & ×0.03 \\ \hline
	\end{tabular}}}
	\caption{Valeurs des indicateurs numériques en fin de simulation.}
	\label{tab:results-basique}
\end{table}
%\footnotetext{Objectif en fin de simulation}