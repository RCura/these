% !TEX root = ../These_Robin_Master.tex
\setcounter{chapter}{0}
\graphicspath{{chap0-2-Intro/}}

\FancyNotChapter{Introduction générale}
\chapter*{Introduction générale}
\label{chap:intro}
%\addcontentsline{toc}{chapter}{Introduction générale}
\mtcaddchapter[Introduction générale]
\vspace*{-5em}
\begin{center}
{\large Version \hl{non définitive du 2019-12-26}}
\end{center}
\vspace*{-1em}
%\begin{itemize}
%	\item 24/11/2019 : début réflexion plan
%	\item 04/12/2019 : version 0 - mise sur Google Docs pour Lena
%	\item 26/12/2019 : version finale
%\end{itemize}



\epigraph{
	\og 
	Modéliser, \textelp{} c'est rechercher quelle composition de modèles rend le mieux compte d'une organisation régionale ou locale, d'une configuration de champ ou de réseau, ou d'une distribution spatiale.
	\textelp{}
	Ces modèles ayant un sens défini et connu, représentant un processus, une réponse sociale à des problèmes spatiaux, ils éclairent ce qui est ou ce qui a été en jeu.
	Tel est le pari central. C'est un pari heuristique.
	La modélisation est une procédure de recherche. 
	Comme telle, elle appelle des règles de rigueur et de bon usage.
	La modélisation en géographie commence par deux questions inévitables et associées : où sommes-nous ? qu'est-ce qui a des chances d'être en jeu ?
	\fg{}
}{\citeauteur{brunet2000modeles} (\citeyear{brunet2000modeles}, p.~28)}

\vspace*{-2.5em}\epigraph{
\og Si un ``objet spatial'', (une organisation, une entité), est observable c'est forcément qu'il est apparu et qu'il s'est maintenu suffisamment longtemps pour être considéré comme étudiable par les géographes.
En conséquence, expliquer cet objet implique qu'on montre à la fois comment il est né et comment il a persisté.
On ne peut donc pas se passer du temps pour faire de la géographie.
Mais le temps dont les géographes doivent se servir est complexe.
Ils ont à se préoccuper non du passé dans son intégralité, mais des traces de ce passé sensibles dans le monde actuel, donc de ``mémoires''
\fg{}.
}{\citeauteur{durand-dastes_jamais_1999} (\citeyear{durand-dastes_jamais_1999}, p.~23)}


\vspace*{-2.5em}\epigraph{
	\og Principe 1 : Modéliser, c'est apprendre\fg{}.
}{\citeauteur{banos_pour_2013} (\citeyear{banos_pour_2013}, p.~77)}

Le \og thème\fg{} de cette thèse n'en est pas un, au sens conventionnel.
Il est relatif à un enjeu méthodologique : comment modéliser des dynamiques spatio-temporelles sur le temps long ?
Les trois citations de l'épigraphe permettent de décomposer ce questionnement générique.
\citeauteur{brunet2000modeles} aide à définir ce qu'est la modélisation en géographie.
Pour \citeauteur{durand-dastes_jamais_1999}, l'explication d'un objet spatial contemporain demande nécessairement de prendre en compte sa dimension temporelle.
Enfin, \citeauteur{banos_pour_2013} nous rappelle l'un des premiers objectifs de toute procédure de modélisation en sciences humaines et sociales.
Ces auteurs, figures majeures de la modélisation dans la géographie française, nous aident à retrouver le \og qu'est-ce\fg{} (la définition), le \og quoi\fg{} (l'\og{}objet spatial\fg{} temporel), et le \og pourquoi\fg{} (le principe heuristique) de telles approches.
Les propositions méthodologiques de mise en œuvre de procédures de modélisation sont nombreuses dans la littérature, mais difficiles à appliquer à un contexte interdisciplinaire de modélisation de l'évolution de systèmes de peuplement sur le temps long.
L'ambition de ce travail de thèse est de proposer et d'expliciter, dans la continuité des auteurs mis en exergue, un nouveau \og comment\fg{}, c'est-à-dire une méthode de modélisation adaptée à un tel contexte scientifique.

Cette thèse décrit, illustre et met en œuvre une approche de modélisation, vue comme \og procédure de recherche\fg{} \autocite{brunet2000modeles}, fondée sur la co-construction interdisciplinaire d'un modèle de simulation.
Tout au long de ce manuscrit, nous cherchons à proposer des méthodes accessibles aux chercheurs en sciences humaines et sociales (SHS), c'est-à-dire des méthodes qui ne requièrent pas, en particulier, de connaissances informatiques ou quantitatives avancées.
Pour cela, nous nous appuyons sur des formalismes partagés par les disciplines des SHS.
Pour la modélisation, il s'agit de l'usage de la simulation à base d'agents qui utilise une analogie anthropomorphique pour décrire les composantes du modèle (les agents), leurs logiques de fonctionnement (les comportements ou \og réflexes\fg{}) et leurs interactions.
Pour analyser, évaluer et guider le développement d'un tel modèle, nous proposons un usage généralisé de la représentation graphique de données.
Cette approche graphique est notamment mobilisée dans le cadre d'une proposition de méthode d'évaluation des sorties de modèle, que nous nommons \og évaluation visuelle\fg{}.

Le cas d'étude sur lequel la démarche prend appui concerne la modélisation des transformations du système de peuplement de l'Europe du Nord-Ouest entre 800 et 1100.
Lors de cette période, l'habitat, initialement majoritairement rural et dispersé, est progressivement polarisé par les églises et les châteaux, formant un système de peuplement hiérarchisé, stable, et territorialisé par le maillage paroissial.
Ce cas d'étude a profondément orienté la démarche proposée dans la présente thèse, impliquant notamment un travail interdisciplinaire pour rendre compte de ces processus non seulement passés, et qui se sont d'autre part déroulés sur le temps long.
Dans la suite de cette introduction, nous présenterons les principaux enjeux que posent un tel cas d'étude.


\subparagraph{Étudier les systèmes de peuplement sur le temps long}~\\
La citation de \textcite{durand-dastes_jamais_1999} rappelle que pour comprendre un \og objet spatial\fg{}, il est nécessaire d'étudier \og comment il est né et comment il a persisté\fg{}.
En nous inscrivant dans la continuité de l'auteur, nous considérons qu'étudier les systèmes de peuplement implique de comprendre les raisons, générales et particulières, de leur état actuel.
En effet, l'organisation de tels systèmes dépend largement des successions de transformations qui les ont façonnés tels qu'on les connaît aujourd'hui.
Pour les systèmes de peuplement anciens, parmi lesquels celui du Nord-Ouest européen, ces successions de transformations ne peuvent être conçues que dans le temps long.
C'est dans cette temporalité que l'on peut observer s'exprimer les transformations sociales successives qui façonnent et transforment l'espace plus ou moins progressivement.
Dès lors, pour décrire et comprendre un système de peuplement ancien dans son présent, il est nécessaire de remonter dans le passé et de saisir les conditions de son émergence.

\textcite{durand-dastes_jamais_1999} ajoute un élément important quant à la manière de mobiliser ce temps long dans le cas des systèmes de peuplement : 	\og [Les géographes] ont à se préoccuper non du passé dans son intégralité, mais des traces de ce passé sensibles dans le monde actuel, donc de “mémoires”\fg{}.
Pour Durand-Dastès, le géographe s'appuie en effet sur des \og traces\fg{} persistantes pour étudier les systèmes.
Pour comprendre la genèse et les transformations d'un système, on ne peut faire l'économie de mobiliser toutes les sortes de traces, c'est-à-dire de sources qui permettent d'éclairer les états passés de ce système.


\subparagraph{De l'étude pluridisciplinaire à l'interdisciplinarité}~\\

Les systèmes de peuplements anciens sont généralement accessibles par le biais de sources rares, incomplètes et lacunaires.
Pour pouvoir étudier un tel système, il est alors nécessaire de compenser la rareté de chaque source par leur démultiplication, en faisant appel à une large quantité et diversité de sources.
Dans le paysage académique français, l'histoire et l'archéologie, disciplines qui étudient toutes deux le passé, sont traditionnellement différenciées par le type de sources qu'elles traitent :
	l'histoire fonde son analyse sur des sources écrites et l'archéologie sur des sources matérielles.
Il est donc intéressant de faire appel conjointement à ces deux disciplines pour multiplier les types de sources analysées afin de maximiser la quantité globale de connaissances expertes fondées sur ces données empiriques.

L'étude d'un système de peuplement ancien et sur le temps long implique alors \textit{a minima} la mise en place de groupes de recherche pluridisciplinaires, composés d'archéologues, d'historiens et de géographes, chacun apportant une expertise dans un type de source ou dans la compréhension des processus qui ont mené aux transformations du système de peuplement.
Cette pluridisciplinarité permet la multiplication des points de vue, chacun analysant avec sa culture disciplinaire les éléments du système dont il a une connaissance experte.
Au-delà de l'expertise propre de chaque chercheur, c'est aussi la multiplication et la diversité des points de vue qui permet d'embrasser une vision plus globale des processus étudiés.

Il est toutefois difficile de parvenir à cette “vision plus globale” tant les \textit{a priori} disciplinaires sont importants.
Ceux-ci peuvent porter sur la manière de définir un objet – la ville des géographes est-elle la même que la ville des archéologues, ou que celle des historiens ? –, mais encore sur un angle spécifique d'explication des processus – interactions spatiales pour les géographes, rôle des institutions et des élites pour les historiens, etc.
Pour dépasser la simple juxtaposition des positions disciplinaires et aboutir à une vision renouvelée, originale, de la description ou de l'explication d'un processus spatio-temporel sur le temps long, il est nécessaire de trouver un point d'intersection entre disciplines et de mettre en commun les objets étudiés et leur analyse, en s'emparant de l'ensemble des éléments propres à chacune d'entre elles.
En adoptant une telle approche, qui n'est plus simplement pluridisciplinaire mais interdisciplinaire, il est possible de dépasser les positionnements disciplinaires classiques, en faveur d'une analyse collectivement validée, par adhésion, proposant une approche plus complète et générique.
L'enjeu est alors d'identifier un objet commun, à l'interface entre les disciplines, suffisamment spécifique pour que chacun puisse se l'approprier avec son bagage disciplinaire et suffisamment générique pour que sa définition puisse être élaborée collectivement et partagée.

\subparagraph{Modélisation}~\\
L'élaboration collective et interdisciplinaire d'un tel objet, pour qu'elle soit satisfaisante pour toutes les disciplines impliquées, peut être facilitée par le recours à un mode de description explicite.
Il s'agit de formaliser la description de cet objet, permettant d'aboutir à la formulation d'une représentation partagée de l'objet, c'est-à-dire un modèle.
L'un des pionniers de l'intelligence artificielle, \textcite[\S 2]{minsky_matter_1965}, donne cette définition très générale des modèles :
\og To an observer B, an object A* is a model of an object A to the	extent that B can use A* to answer questions that interest him about A.\fg{}.
La formalisation d'un modèle, quel qu'en soit le paradigme, permet de répondre à des questions.
Dans notre cas, c'est en exprimant, sous la forme d'un modèle formalisé, nos hypothèses thématiques, sur la transformation du système de peuplement Nord-Ouest européen entre 800 et 1100, que nous facilitons la recherche des explications de cette transformation, et surtout, que nous pouvons y attribuer un caractère générique.

Dans un contexte d'interdisciplinarité, le recours à un modèle offre un autre avantage : 
	expliciter le sens exact, ontologique, des entités du modèle et les éventuelles relations qui les mettent en interaction grâce à la formalisation mobilisée dans la construction de ce modèle.
En effet, pour que plusieurs chercheurs puissent répondre à la même question sur un système (A) à partir d'un unique modèle (A*), il est indispensable qu'ils s'accordent sur la définition précise du contenu de A*, et donc qu'ils forment un « observateur » collectif cohérent B\footnote{
	Pour \textcite[\S 2]{minsky_matter_1965}, on ne peut en effet utiliser A* si B n'est pas clairement identifié et compris : \og Any attempt to suppress the role of the intentions of the investigator B leads to circular definitions or to ambiguities about ``essential features'' and the like.\fg{}
}.
Pour que des chercheurs, d'origines disciplinaires variées, et ayant différentes approches, puissent parvenir à une représentation commune -- un modèle, il est nécessaire qu'ils s'accordent sur sa définition\footnote{
	\textcite[22]{brunet2000modeles} explique ainsi que \og toute définition d'un objet propose un modèle\fg{}.
}.
Il s'agit de choisir les objets à y inclure, de circonscrire la relation de correspondance entre ces objets construits et le monde observable (domaine empirique), et d'harmoniser le sens précis de ce que sont les relations : \og qu'est-ce qui a des chances d'être en jeu ?\fg{} pour reprendre les mots de  \textcite{brunet2000modeles}.

L'exercice de la modélisation -- la création d'un modèle -- est d'autant plus intéressant qu'il a une forte dimension heuristique.
Le premier principe de \textcite{banos_pour_2013}, \og modéliser, c'est apprendre\fg{}\footnote{
	Ce principe deviendra même le titre de la publication correspondant à l'habilitation à diriger les recherches de cet auteur \autocite{banos_modeliser_2016}.
	
}, souligne que quelle que soit la forme choisie\footnote{
	\textcite[p.27, \S D]{brunet2000modeles} en identifie trois : \og rhétorique\fg{}, \og mathématique\fg{} et \og iconique\fg{} (qu'il préfère au terme graphique, bien plus usité).	
}, la synthèse des connaissances sous la forme d'un modèle force le chercheur à conceptualiser l'objet de sa recherche et à en repenser le fonctionnement.
Pour \textcite[p.28, \S F]{brunet2000modeles}, la modélisation \og  construit, déconstruit et reconstruit ; elle passe par une série d'itérations entre déduction et induction\fg{}.
C'est au moyen de ces itérations que le modélisateur consolide graduellement ses connaissances des processus et devient capable de les exprimer sous la forme d'un modèle descriptif et explicatif.
En contexte interdisciplinaire, l'aspect heuristique de la modélisation est renforcé par la confrontation des connaissances hétérogènes et des points de vue de chacun des membres impliqués dans la modélisation.
Le modèle prendra en effet appui sur les connaissances, complémentaires, de chacun.
Dès lors, en tant que processus de recherche, la modélisation aboutit indubitablement à un enrichissement des connaissances et des représentations de chacun grâce à celles des autres.

\subparagraph{Simulation}~\\
Un modèle, comme toute construction scientifique, a vocation à être éprouvé.
Pour les modèles portant sur des phénomènes sociaux ou spatiaux contemporains, une telle mise à l'épreuve peut être menée de manière empirique, via des expériences contrôlées \og en laboratoire\fg{}, ou par des observations des phénomènes \textit{in situ}, \og sur le terrain\fg{}. 
Dans notre cas, c'est-à-dire celui de la modélisation d'un processus spatial ancien et sur le temps long, il est impossible de mener une telle expérience de manière contrôlée ou même \textit{in situ}.
Pour pallier cette impossibilité, la simulation informatique constitue une solution particulièrement intéressante, en ce qu'elle permet de mettre en place un \og laboratoire virtuel\fg{}, autrement dit de mener des expériences \textit{in silico}.
En effet, la re-formalisation computationnelle (algorithmique ou mathématique) du modèle exprimé préalablement offre la possibilité d'en simuler le déroulement et l'aboutissement, et donc d'évaluer le modèle, c'est-à-dire de comparer ses sorties avec les connaissances empiriques du phénomène modélisé.

La simulation permet de tester un modèle, mais aussi d'y ajouter le temps en le dotant d'une dimension temporelle quasi-continue, contrairement à un modèle \og rhétorique\fg{} ou \og iconique\fg{}\footnote{
	Certains modèles \og iconiques\fg{} peuvent comporter une dimension temporelle, comme les modèles de \og chrono-chorématique\fg{} \autocite{boissavit2005chrono,rodier2010dossier}.
	La simulation permet de rendre continu le temps discret de ces modèles graphiques, ou de réduire l'intervalle de temps entre les instantanés, c'est-à-dire, dans tous les cas, d'en augmenter la granularité temporelle. 
}.
La simulation est intrinsèquement dynamique et la dimension temporelle peut être observée pendant son déroulement, et ce selon différentes granularités.
Avec la simulation, on peut évaluer un état final, mais aussi la succession d'instantanés formés par les états intermédiaires par lesquels passe le système.
Un modèle conceptuel permet de détailler les règles de fonctionnement d'un système, lesquelles peuvent mener à une transformation de ce système.
Avec un modèle de simulation, on peut directement observer, et donc analyser, ces transformations.
Ainsi, il devient possible d'évaluer la capacité d'un modèle à reproduire un état connu, mais aussi de vérifier la cohérence des transformations pendant qu'elles se produisent au regard des connaissances empiriques.


\subparagraph{Le choix du paradigme de la simulation multi-agents}~\\

La conception d'un modèle \og rhétorique\fg{} ou \og iconique\fg{} est relativement accessible à chacun. 
En effet, les formalismes employés -- l'écrit et le visuel -- ont une place centrale dans nos sociétés, quand bien même ils reposent aussi sur des apprentissages de construits sociaux partagés.
En plus de cela, la construction d'un modèle mathématique ou informatique requiert une culture spécifique.
Cela passe notamment par l'apprentissage de différents \og langages\fg{} et paradigmes, qui peuvent être mathématiques (systèmes d'équations différentielles, formalismes de la théorie des jeux, logique mathématique, etc.) ou informatiques (programmation orientée objet, programmation fonctionnelle, spécificité de chaque langage informatique, etc.).
La construction de modèles de simulation demande donc une certaine maîtrise d'un langage informatique et, à ce titre, requiert un coût d'entrée important qui la rend moins accessible que d'autres formes de modélisation.
Pour créer un modèle de simulation, il est ainsi nécessaire de disposer des ces compétences mathématiques ou informatiques.
Dans le cadre d'un projet collectif, cela implique que tous les participants aient des connaissances méthodologiques générales sur la modélisation, et qu'au moins un des participants possède, en plus, les compétences techniques requises à l'expression dans un langage formel.

Chez les chercheurs en SHS, ces compétences mathématiques et informatiques ne font pas partie d'un socle commun et partagé de connaissances, et notamment chez les archéologues et historiens qui s'intéressent à la modélisation des systèmes de peuplement.
Parmi les types formels de modélisation, il nous semble que le paradigme, informatique, de la modélisation à base d'agents, aussi dénommé \og simulation multi-agents\fg{} (SMA), est plus accessible que les autres.
En effet, en SMA, on décrit le fonctionnement d'un système par le biais d'agents informatiques, représentation d'acteurs individuels, collectifs ou institutionnels empiriques.
Ces agents interagissent les uns avec les autres au moyen de \og règles de comportement\fg{} que l'on nomme des mécanismes.
Chaque mécanisme est précisément défini par les modélisateurs, mais le résultat des interactions qui en émergent ne peut être déterminé en amont.
Le parallèle avec les comportements individuels, rationnels ou non, des acteurs des sociétés passées (paysans, seigneurs, etc.), ou des agrégats d'acteurs (villages, communautés, etc.) permet ainsi d'expliciter le fonctionnement d'un modèle.
Ce type de modélisation, par sa nature anthropomorphique, nous semble plus facilement appropriable intellectuellement pour des chercheurs en SHS que des formalismes moins assimilables à des comportements individuels empiriques.

\subparagraph{Une modélisation en co-construction}~\\

Si le paradigme des SMA est plus accessible aux chercheurs en SHS que d'autres, il demeure un formalisme informatique où l'on exprime les entités sous formes d'objets informatiques dotés de règles, c'est-à-dire de fonctions.
L'objectif est que les chercheurs impliqués, de différentes disciplines et de niveaux de connaissances informatiques variés, aient une compréhension du modèle qui soit commune.
Cette compréhension ne peut être totalement similaire du fait des arrières plans disciplinaires, mais il s'agit de s'en approcher le plus possible par une explicitation fine.
Il faut alors mettre en place une démarche collective et ainsi veiller à ce que chaque participant au projet puisse s'approprier le modèle.
Pour cela, nous avons fait le choix de la \og co-construction\fg{}, c'est-à-dire d'une expérience collective où chacun participe à l'élaboration de l'ensemble des composants du modèle, depuis les aspects les plus conceptuels (choix des entités) jusqu'aux aspects les plus détaillés (ordre d'appel des mécanismes).

Ce faisant, chacun participe à l'ensemble des réflexions et décisions liées au modèle.
Toutes les étapes de construction du modèle résultent d'un compromis entre les participants, ce qui permet de dépasser les \textit{a priori} disciplinaires et les \og rôles\fg{} de chacun dans le collectif.
Une telle démarche renforce aussi la dimension heuristique du modèle car chacun doit s'exprimer dans un formalisme commun et partagé :
	un programme informatique n'interprète pas mais exécute, et ne laisse ainsi pas de part au flou.
Pour que le modèle de simulation soit exécutable\footnote{
	On peut parler de \og compilable\fg{} en vocabulaire informatique.
}, il faut une certaine forme d'exhaustivité dans l'explicitation.
La mise en œuvre effective d'un processus de co-construction est de ce fait facilitée par l'implémentation informatique d'un modèle.
Même sans que cette implémentation soit entièrement collective, c'est-à-dire que chacun participe par exemple à l'écriture des lignes de code du modèle, l'emploi du formalisme informatique forcera chacun à expliciter le détail de l'implémentation.

\subparagraph{Des outils pour faciliter la co-construction}~\\

Pour guider le développement et l'amélioration du modèle, il est utile de pouvoir en analyser le comportement, c'est-à-dire de procéder à son évaluation.
Il s'agit de se demander si le modèle correspond entièrement aux attentes des modélisateurs.
L'évaluation de modèles est un domaine de recherche à part entière, souvent fondé sur des analyses formelles, statistiques ou du moins quantifiées des sorties du modèle. 
Évaluer un modèle selon ces méthodes requiert, d'une part, une certaine expertise thématique sur le phénomène qui est modélisé, et, d'autre part, une bonne connaissance des méthodes d'analyse quantitative des données.
Comme la connaissance de la modélisation mathématique ou informatique, ces méthodes quantitatives ne font pas forcément partie de l'outillage méthodologique classique en SHS, et notamment en histoire et en archéologie.

Pour faciliter l'évaluation du modèle, nous proposons dans cette thèse une approche, intitulée \og évaluation visuelle\fg{}, fondée sur l'analyse graphique des sorties du modèle.
Cette approche permet à chaque participant du groupe interdisciplinaire d'évaluer, de manière qualitative, les sorties du modèle, y compris quand celles-ci sont massives.
L'évaluation visuelle consiste en l'analyse visuelle d'un ensemble d'\og{}indicateurs de sorties\fg{} et de représentations graphiques des différentes dynamiques générées par le modèle de simulation.
Cette approche rend possible l'exercice d'un regard critique sur ces sorties au regard d'une grille d'analyse explicitée \textit{a priori}.
La multiplication et la diversification des indicateurs de sortie permet d'obtenir une évaluation plus fine et complète du modèle et rend accessible l'évaluation de sous-ensembles du modèle (agents, mécanismes).
Ce sont donc la couverture globale et la finesse de résolution de l'évaluation qui sont accrues avec cette approche.
En facilitant et en fluidifiant l'analyse, globale ou fine, cette méthode ouvre aussi la voie à l'amélioration du modèle, en permettant, dans une démarche faite d'allers-retours, d'isoler rapidement les composantes les moins satisfaisantes du modèle.

Pour rendre ce processus d'évaluation facilement accessible en donnant à chacun la capacité d'explorer les sorties du modèle et de juger de leur pertinence, une plateforme interactive, appelée \og SimEDB\fg{}, a été développée.
Celle-ci permet aux membres du projet de visualiser eux-mêmes les sorties du modèle sans avoir à acquérir de compétences spécifiques en analyse et en visualisation de données.
En explorant ces sorties graphiques, les chercheurs impliqués peuvent comprendre comment les dynamiques et les structures sociales et spatiales émergent des interactions entre les agents du modèle.
Ces chercheurs peuvent ainsi tester des hypothèses thématiques liées au phénomène qu'ils souhaitent étudier avec le modèle.
En cela, l'évaluation visuelle au moyen de la plateforme d'exploration favorise une démarche abductive, guidée par les données, et accroît la dimension heuristique du processus de modélisation.


\subparagraph{Organisation du manuscrit}~\\
Présenter un modèle complexe, sa démarche de construction et son évaluation, dans le cadre d'un manuscrit de thèse, revêt forcément une certaine artificialité dans l'ordre de présentation.
Il s'agit de rendre linéaire et chronologique un processus profondément \og en spirale\fg{}, constitué de nombreux allers-retours.
De même, la présentation d'un unique état de l'art initial, pour une thèse dont les inscriptions thématiques et méthodologiques sont multiples, n'apparaît pas comme l'option la plus adaptée.
Ce manuscrit présente ainsi de nombreux renvois entre chapitres (précédents et suivants), ainsi que des états de l'art propres à chaque chapitre.

Le premier chapitre (\cref{chap:chap1}) introduit et précise le positionnement adopté dans ce travail.
Il s'agit d'expliciter les raisons qui nous ont amené à choisir la simulation multi-agents pour modéliser le phénomène étudié, à construire une plateforme consacrée à la représentation graphique, ainsi qu'à développer une approche de co-construction.
Le \cref{chap:chap2} présente le modèle \og SimFeodal\fg{} qui est issu de cette approche. Le cadre de conception et d'application y est détaillé selon le protocole de présentation de modèles à base d'agents \og ODD\fg{} \autocite{grimm_odd_2010}.
Le \cref{chap:chap3} introduit et définit la méthode d'\og{}évaluation visuelle\fg{}.
On y explicite le choix de cette méthode comme méthode d'évaluation et d'exploration du modèle, en la replaçant dans le cadre plus large de l'évaluation des modèles de simulation.
Cette évaluation visuelle est ensuite mobilisée pour présenter l'évolution du modèle et le travail de \og paramétrage\fg{} qui a guidé cette évolution.
Dans le \cref{chap:chap4}, nous présentons la succession de besoins, de contraintes et de choix -- thématiques, méthodologiques et techniques -- qui a mené à la création de la plateforme interactive d'exploration des sorties du modèle SimEDB.
Les parti-pris de sa conception en matière de stockage de données (bases de données relationnelles analytiques) et d'interactivité sont expliqués dans ce chapitre.
En mobilisant cette plateforme, le \cref{chap:chap5} expose les résultats du modèle, à savoir les sorties de la version la plus récente de SimFeodal, et les critique au regard de la grille d'analyse établie préalablement.
Pour prolonger l'exploration du modèle, ce chapitre est aussi l'occasion de mener une analyse de sensibilité exploratoire du modèle, qui reste fondée sur la représentation graphique.
Enfin, le \cref{chap:chap6} constitue un retour réflexif, à partir de l'expérience relatée tout au long de cette thèse, sur la manipulation de données de simulation et sur l'effectivité de la démarche de co-construction mise en œuvre et relatée tout au long de cette thèse.


\clearpage
\FancyChapter