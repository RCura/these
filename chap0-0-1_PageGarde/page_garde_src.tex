%%%%%%%%%%%%%%%%%%%%%%%%%%%%%%%%%%%%%%%%
%    Page de garde (Pagedegarde.tex)   %
%%%%%%%%%%%%%%%%%%%%%%%%%%%%%%%%%%%%%%%%
% Dorian Depriester, 2014

\makeatletter
\def\@ecole{école}
\newcommand{\ecole}[1]{
	\def\@ecole{#1}
}

\def\@specialite{Spécialité}
\newcommand{\specialite}[1]{
	\def\@specialite{#1}
}

\def\@ED{\'{E}cole Doctorale}
\newcommand{\ED}[1]{
	\def\@ED{#1}
}

\def\@doctorat{Doctorat}
\newcommand{\doctorat}[1]{
	\def\@doctorat{#1}
}

\def\@subtitle{Sous-titre}
\newcommand{\subtitle}[1]{
	\def\@subtitle{#1}
}

\def\@adresse{Adresse}
\newcommand{\adresse}[1]{
	\def\@adresse{#1}
}

\def\@directeur{directeur}
\newcommand{\directeur}[1]{
	\def\@directeur{#1}
}

\def\@encadrant{encadrant}
\newcommand{\encadrant}[1]{
	\def\@encadrant{#1}
}
\def\@jurya{}{}{}
\newcommand{\jurya}[3]{
	\def\@jurya{#1	& #2	& #3\\}
}
\def\@juryb{}{}{}
\newcommand{\juryb}[3]{
	\def\@juryb{#1	& #2	& #3\\}
}
\def\@juryc{}{}{}
\newcommand{\juryc}[3]{
	\def\@juryc{#1	& #2	& #3\\}
}
\def\@juryd{}{}{}
\newcommand{\juryd}[3]{
	\def\@juryd{#1	& #2	& #3\\}
}
\def\@jurye{}{}{}
\newcommand{\jurye}[3]{
	\def\@jurye{#1	& #2	& #3\\}
}
\def\@juryf{}{}{}
\newcommand{\juryf}[3]{
	\def\@juryf{#1	& #2	& #3\\}
}
\def\@juryg{}{}{}
\newcommand{\juryg}[3]{
	\def\@juryg{#1	& #2	& #3\\}
}
\makeatother

\newcommand\BackgroundPic{%
	\put(0,0){%
		\parbox[b][\paperheight]{\paperwidth}{%
			\includegraphics[height=0.45\paperheight]{bordure.png}%
			\vfill
		}
	}
}

\makeatletter
\newcommand{\pagedegarde}{
	\newgeometry{top=1.5cm, bottom=1cm, left=1cm, right=1cm}
	%\AddToShipoutPicture*{\BackgroundPic}
	%\AddToShipoutPicture*{\EtiquetteThese}
	\begin{titlepage}
		\centering
		{\large \@ED}\\
		\vspace{1cm}
		{\Large {\bfseries \@doctorat}}\\
		\vspace{.25cm}
		{\large pour obtenir le grade de docteur délivré par}\\
		\vspace{.25cm}
		{\large \@ecole}\\
		\vspace{0.25cm}
		{\large Discipline : \@specialite}\\
		\vspace{1cm}
		\setstretch{2}
		%\title{Modéliser des systèmes de peuplement en interdisciplinarité.}
		%\subtitle{Co-construction et exploration visuelle d’un modèle de simulation}
		{\huge \textbf{Modéliser des systèmes de}}\\
		{\huge \textbf{peuplement en interdisciplinarité.}}\\
		\vspace{1em}
		{\huge \textbf{Co-construction et exploration visuelle}}\\
		{\huge \textbf{d’un modèle de simulation.}}
%		{\LARGE \bfseries{\@title}} \\
%		{\LARGE \bfseries{\@title}} \\
%		{\LARGE \bfseries{\@title}} \\
%		{\LARGE \bfseries{\@subtitle}}\\
		\vfill
		présentée et soutenue publiquement par\\
		\vspace{0.25cm}
		{\LARGE {\bfseries \@author}} \\
		%\vspace{0.25cm}
		le \@date \\
		\vspace{0.5cm}
		\setstretch{1}
		\vspace{.5cm}
		Sous la direction de {\bfseries \@directeur}\\
		\vspace{.5cm}
		%Co-encadrant de thèse : {\bfseries \@encadrant}\\
		\vfill
		\setstretch{.75}
		\begin{tabular}{>{\bfseries}lll}
			\large Composition du Jury : \\
			\@jurya
			\@juryb
			\@juryc
			\@juryd
			\@jurye
			\@juryf
			\@juryg
		\end{tabular}
		\vfill
		\includegraphics[height=4em]{chap0-0-1_PageGarde/logo_P1.png}
		\hspace{1cm}
		\includegraphics[height=5em]{chap0-0-1_PageGarde/logo_GeoCites.png}
		\hspace{2cm}
		\includegraphics[height=5em]{chap0-0-1_PageGarde/logo_LabEx.png}
		\hspace{2cm}
		\includegraphics[height=4em]{chap0-0-1_PageGarde/logo_ED.png}
		
		%	\@adresse
		\pagestyle{empty}
		\cleardoublepage
	\end{titlepage}
	
	\restoregeometry  
}