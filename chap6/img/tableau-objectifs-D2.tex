\begin{table}[H]
	\captionsetup{singlelinecheck=off}
	\centering
	\small
	\resizebox{\textwidth}{!}{%
	{\renewcommand{\arraystretch}{1.1}%
		\begin{tabular}{|p{3.25cm}|p{2.1cm}|p{2cm}|p{1.75cm}|p{1.75cm}|}
			\hline
			\textbf{Indicateur} & \textbf{Valeur} \textbf{attendue} & \textbf{Valeur de référence} & \textbf{Moyenne} & \textbf{Médiane} \\ \hline
			\textit{Agrégats} & \textit{200} & 249 & 198 & 199 \\ \hline
			\textit{Gros châteaux} & \textit{10} & 15 & 12 & 12 \\ \hline
			\textit{Églises paroissiales} & \textit{300} & 348 & 252 & 253 \\ \hline
			\textit{Distance moyenne entre églises} & \textit{3 000 m} & 1 459 m & 1 988 m & 1 975 m \\ \hline
			\textit{Part de foyers paysans isolés} & \textit{20 \%} & 30 \% & 21 \% & 21 \% \\ \hline
			\textit{Augmentation de la charge fiscale des foyers paysans} & \textit{x 3} & x 2.4 & x 2.6 & x 2.6 \\ \hline
	\end{tabular}}}
	\caption[Valeurs des indicateurs numériques du scénario D2 en fin de simulation.]{Valeurs des indicateurs numériques du scénario D2 en fin de simulation.
	\small
	\textit{Les valeurs de référence correspondent aux moyennes obtenues dans la version calibrée de SimFeodal, sans croissance démographique donc.}}
	\label{tab:results-D2}
\end{table}