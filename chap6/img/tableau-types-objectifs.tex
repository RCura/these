\begin{table}[H]
	\centering
	\small
	\resizebox{\textwidth}{!}{%
	{\renewcommand{\arraystretch}{1.2}%
	\begin{tabular}{|p{4.5cm}|p{2.1cm}|p{1.75cm}|p{4.5cm}|}
		\hline
		\textbf{Indicateur de sortie de simulation} & \textbf{Valeur attendue} & \textbf{Type} & \textbf{Dépendances directes} \\ \hline
		\textit{Nombre d'agrégats} & \textit{200} & Émergent & \multicolumn{1}{c|}{--} \\ \hline
		\rowcolor[HTML]{DCDCDC} \textit{Nombre de châteaux} & \textit{50} & Contextuel & probabilités de construction de châteaux (PS et GS) \\ \hline
		\textit{Nombre de gros châteaux} & \textit{10} & Émergent & \multicolumn{1}{c|}{--} \\ \hline
		\rowcolor[HTML]{DCDCDC} \textit{Nombre de seigneurs} & \textit{200} & Contextuel & paramètre~dédié~: (\textsf{objectif\_nombre\_seigneurs}) \\ \hline
		\textit{Nombre d'églises paroissiales} & \textit{300} & Émergent & \multicolumn{1}{c|}{--} \\ \hline
		\textit{Distance moyenne entre églises} & \textit{3 000 m} & Émergent & \multicolumn{1}{c|}{--} \\ \hline
		\textit{Part de foyers paysans isolés} & \textit{20 \%} & Émergent & \multicolumn{1}{c|}{--} \\ \hline
		\textit{Augmentation de la charge fiscale des foyers paysans} & \textit{x 3} & Émergent & \multicolumn{1}{c|}{--} \\ \hline
		\rowcolor[HTML]{DCDCDC} \textit{Nombre de Foyers Paysans} & 50 000 & Contextuel & nombre initial; taux de croissance\\ \hline
		\rowcolor[HTML]{DCDCDC} \textit{Densité de population} & 8 feux/km² & Contextuel & nombre de foyers paysans ; taille du monde \\ \hline
	\end{tabular}}}
	\caption[Les indicateurs de sortie de simulation quantitatifs de SimFeodal]{%
		Les indicateurs de sortie de simulation quantitatifs de SimFeodal.
		\textit{Les lignées grisées désignent les indicateurs \og contextuels\fg{}.}
	}
	\label{tab:objectifs-types}
\end{table}