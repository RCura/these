%\chapter{Les apports de la visualisation dans l'exploration d'un modèle}
%\label{chap:chap6}
%\begin{center}
%	{\large Version 2018-XX-XX}
%\end{center}
%\minitoc
%
%\section{Comment visualiser des données de simulation?}
%\subsection{Intégrer les données continuellement pour favoriser les allers-retours}
%\subsection{Visualisation et agrégation}
%\subsection{Quelle(s) agrégation(s) d'un espace théorique ?}
%\subsection{Visualiser les variations}
%
%\section{De la visualisation à l'exploration interactive}
%\subsection{Les apports des \textit{visual analytics} à la compréhension des données}
%\subsection{Rendre plus accessibles des données complexes et massives}
%\subsection{Pousser à la sérendipité par l'exploration interactive et intuitive}
%
%\section{Co-évolution du modèle et de ses interfaces d'exploration}
%\subsection{Adapter les outils aux demandes des utilisateurs}
%\subsection{Adapter les outils aux évolutions du modèle}
%\subsection{Comment comparer des modèles dotés d'indicateurs différents ?}


\chapter{Exploration du comportement de SimFeodal}
\label{chap:chap6}
\begin{center}
	{\large Version 2019-07-05}
\end{center}
\minitoc

\section{Calibrage du modèle et premiers résultats}
\subsection{Quels objectifs}
\subsection{Phases de calibrage : sur quoi a-t-on joué}
\subsection{Les \og résultats\fg{} de SimFeodal}

\section{Analyser la sensibilité de SimFeodal}
\subsection{Analyse de sensibilité - Méthodo}
\subsection{Analyse de sensibilité - Résultats (quanti))}
\subsection{Analyse de sensibilité - Évaluation visuelle}
\subsection{Analyser la sensibilité à l'aléa}

\section{Comprendre le modèle par l'exécution de scénarios}
\subsection{Tester l'hypothèse d'une croissance démographique}
\subsection{Modéliser la dépendance spatiale : le poids du servage}
\subsection{Quel rôle et importance des communautés paysannes dans la structuration du système de peuplement ?}